%%%%%%%%%%%%%%%%%%%%%%%%%%%%%%%
%%%%%%%%%%%%%%%%%%%%%%%%%%%%%%
%%
%%     INTRO
%%		go down...
%%%%%%%%%%%%%%%%%%%%%%%%%%%%%%
%%%%%%%%%%%%%%%%%%%%%%%%%%%%%%
\documentclass[11pt]{report}
\usepackage{amssymb,amsmath,amsthm,amsfonts}
\usepackage{multicol,multirow}
\usepackage{calc}
\usepackage{ifthen}
\usepackage[a4paper]{geometry}
\usepackage{graphicx}
\usepackage{bm}
\usepackage{multirow}
\usepackage{mathrsfs}
\usepackage[absolute]{textpos}
\usepackage{tabularx}
\usepackage{undertilde}
\usepackage{titlesec}
\usepackage{hyperref}
\usepackage[inline]{enumitem}
\usepackage[utf8]{inputenc}
\usepackage{wrapfig}

\titleformat{\chapter}[hang]   
{\normalfont\huge\bfseries}{\thechapter\quad}{0cm}{}  
\titlespacing*{\chapter}{0cm}{0cm}{0cm}
\titlespacing*{\section}{0cm}{.3cm}{0cm}
\titlespacing*{\subsection}{.2cm}{.1cm}{-.1cm}
\titlespacing*{\subsubsection}{.35cm}{0cm}{-.1cm}

\newcommand{\sz}[1]{\scriptsize #1\normalsize}
\newcommand{\ul}[1]{\underline{\smash{#1}}}
\newcommand{\de}{\mathrm d }
\newcommand{\ma}{\mathrm{Ma}}

\hypersetup{linktoc=all}

\geometry{top=.1cm,left=.5cm,right=.5cm,bottom=.1cm}

\begin{document}
	\tableofcontents
	\newpage
		
\chapter{Equazioni del moto}
	\section{Equazioni del moto}
	\underline{Def:} traduzione in termini matematici dei principi di conservazione della massa, della quantità di moto e dell'energia\\
	\textit{Oss:} usando il th. del trasporto ottengo le varie forme integrali delle equazioni di bilancio. Applicando in seguito il lemma di localizzazione ottengo le rispettive forme differenziali.
	
	\section{Teorema del trasporto}
	\underline{Def:} \begin{tabularx}{\linewidth}{|l|X|X|}
	\hline
    &forma conservativa&forma non conservativa\\
    \hline
	mista&$\int_{\Omega(t)}[\frac{D\vec F}{Dt}+\vec F\,\mathrm{div}\,\vec V]\,\de \upsilon$&$\int_{\Omega(t)}\rho\frac{D\vec f}{Dt}\,\de \upsilon$\\
	euleriana&$\int_{\Omega(t)}[\frac{\partial\vec F}{\partial t}+\mathrm{div}(\vec F\!\otimes\!\vec V)]\,\de \upsilon$&$\int_{\Omega(t)}\rho(\frac{\partial\vec f}{\partial t}+\vec V\cdot\vec\nabla f)\,\de \upsilon$\\
  	\hline
    \end{tabularx}\\
	\textit{Dim:} $\vec F\equiv \vec F(P,t)\;\;\vec{\widetilde F}\equiv\vec{\widetilde F}(\widetilde P,t_0)\quad D_t\int_{\Omega(t)}F\,\de\upsilon
	\overset{1}{=}
	D_t\int_{\widetilde\Omega}\vec{\widetilde F}J\,\de \widetilde\upsilon
	=\int_{\widetilde\Omega}\frac{D(\vec{\widetilde F}J)}{Dt}\,\de \widetilde\upsilon
	=\int_{\tilde\Omega}[\frac{D\vec{\widetilde F}}{Dt}J+\vec{\widetilde F}\frac{DJ}{Dt}]\,\de \tilde\upsilon
	\overset{2}{=}$\\$=
	\int_{\tilde\Omega}[\frac{D\vec{\widetilde F}}{Dt}J+\vec{\widetilde F}J\,\mathrm{div}\,\vec V]\,\de \tilde\upsilon
	=\int_{\Omega(t)}[\frac{D\vec F}{Dt}+\vec F\,\mathrm{div}\,\vec V]\,\de \upsilon$\qquad(forma conservativa mista)\\
	$=\int_{\Omega(t)}[\frac{\partial\vec F}{\partial t}+\vec V\cdot\utilde\nabla\vec F+\vec F\,\mathrm{div}\vec V]\,\de \upsilon
	=\int_{\Omega(t)}[\frac{\partial\vec F}{\partial t}+\mathrm{div}(\vec F\!\otimes\!\vec V)]\,\de \upsilon$\qquad(forma conservativa euleriana)\\
	$=\int_{\Omega(t)}\frac{\partial\vec F}{\partial t}\,\de \upsilon+\int_{\Omega(t)}\mathrm{div}(\vec F\!\otimes\!\vec V)\,\de \upsilon
	=\int_{\Omega(t)}\frac{\partial\vec F}{\partial t}\,\de \upsilon+\int_{\partial\Omega(t)}(\vec F\!\otimes\!\vec V)\hat n\,\de S$\qquad(forma convettiva)\\
	Se $\vec F=\rho\vec f$ allora si può sostituire nella forma mista $\int_{\Omega(t)}[\frac{D(\rho\vec f)}{Dt}+\rho\vec f\mathrm{div}\,\vec V]\,\de \upsilon$ usando la relazione $D_t\rho+\rho\,\mathrm{div}\,\vec V=0\\
	=\int_{\Omega(t)}[\rho\frac{D\vec f}{Dt}+\vec fD_t\rho+\rho\vec f\,\mathrm{div}\,\vec V]\,\de \upsilon
	=\int_{\Omega(t)}[\rho\frac{D\vec f}{Dt}+\vec f(D_t\rho+\rho\,\mathrm{div}\,\vec V)]\,\de \upsilon
	=\int_{\Omega(t)}\rho\frac{D\vec f}{Dt}\,\de \upsilon$\qquad(forma non conservativa mista)\\
	$=\int_{\Omega(t)}\rho(\frac{\partial\vec f}{\partial t}+\vec V\cdot\utilde\nabla\vec f)\,\de \upsilon$\qquad (forma non conservativa euleriana)\\
	\textit{Oss:} la variazione temporale di una grandezza valutata per un volume variabile nel tempo tiene contro, oltre della variazione temporale in ogni punto interno al volume, anche della variazione associata al flusso di massa attraverso la superficie che lo racchiude

	\section{Bilancio della massa} \label{bm}
	\underline{Def:} \begin{tabularx}{\linewidth}{|l|X|}
	\hline
    mista&$D_t\rho+\rho\,\mathrm{div}\vec V=0$\quad\;\scriptsize{(BM1)}\\
	euleriana&$\partial_t\rho+\mathrm{div}(\rho\vec V)=0$\quad\;\scriptsize{(BM2)}\\
	\hline
    \end{tabularx}\\
	\textit{Dim:} $D_t\{m(\Omega(t))\}=0\;\;\Rightarrow\;\;
	\int_{\Omega(t)}[D_t\rho+\rho\,\mathrm{div}\vec V]\,\de \Omega
	=\int_{\Omega(t)}[\partial_t\rho+\mathrm{div}(\rho\vec V)]\,\de \Omega=0$
	
	\section{Bilancio della quantità di moto}	\label{bq}
	\underline{Def:} \begin{tabularx}{\linewidth}{|l|X|X|}
	\hline
    &forma conservativa&forma non conservativa\\
    \hline
	mista&$D_t(\rho\vec V)+\rho\vec V\mathrm{div}\vec V=\rho\vec f+\mathrm{div}\,\utilde T$\quad\;\scriptsize{(BQ1)}&$D_t\vec V=\vec f+\frac 1\rho\,\mathrm{div}\,\utilde T$\quad\;\scriptsize{(BQ3)}\\
	euleriana&$\partial_t(\rho\vec V)+\mathrm{div}(\rho\vec V\!\otimes\!\vec V)=\rho\vec f+\mathrm{div}\,\utilde T$\quad\;\scriptsize{(BQ2)}&$\partial_t\vec V+\vec V\cdot\utilde\nabla\vec V=\vec f+\frac 1\rho\,\mathrm{div}\,\utilde T$\quad\;\scriptsize{(BQ4)}\\
	\hline
    \end{tabularx}\\
	\textit{Dim:} $\vec F_\mathrm{vol}+\vec F_\mathrm{sup}=D_t\vec Q\;\;\Rightarrow\;\;
	\int_{\Omega(t)}\rho\vec f\,\de \Omega
	+\int_{\partial\Omega(t)}\utilde T\hat n\,\mathrm  d(\partial\Omega)
	=D_t\int_{\Omega(t)}\rho\vec V\,\de \Omega$\\
	Applico il TdD al primo membro\quad$\int_{\Omega(t)}[\rho\vec f+\mathrm{div}\,\utilde T]\,\de \Omega
	=D_t\int_{\Omega(t)}\rho\vec V\,\de \Omega$\\
	Per le forme conservative\quad$D_t\int_{\Omega(t)}\rho\vec V\,\de \Omega
	=\int_{\Omega(t)}[D_t(\rho\vec V)+\rho\vec V\mathrm{div}\vec V]\,\de \Omega
	=\int_{\Omega(t)}[\partial_t(\rho\vec V)+\mathrm{div}(\rho\vec V\otimes\vec V)]\,\de \Omega$\\
	Per le forme non conservative\quad$D_t\int_{\Omega(t)}\rho\vec V\,\de \Omega
	=\int_{\Omega(t)}\rho D_t\vec V\,\de \Omega=\int_{\Omega(t)}\rho(\partial_t\vec V+\vec V\cdot\utilde\nabla\vec V)\,\de \Omega$
	 
	\section{Bilancio dell'energia totale}	\label{be}
	\underline{Def:} \begin{tabularx}{\linewidth}{|l|X|X|}
	\hline
    &forma conservativa&forma non conservativa\\
    \hline
	mista&$D_t\{\rho(e+\frac{V^2}{2})\}+\rho(e+\frac{V^2}{2})\mathrm{div}\,\vec V=\rho\vec f\cdot\vec V+\ldots$\quad\;\scriptsize{(BE1)}&$D_t\{e+\frac{V^2}{2}\}=\vec f\cdot\vec V+\frac 1\rho\,\mathrm{div}(-\vec q+\utilde T\vec V)$\quad\;\scriptsize{(BE3)}\\
	euleriana&$\partial_t\{\rho(e+\frac{V^2}{2})\}+\mathrm{div}(\rho(e+\frac{V^2}{2})\vec V)\!=\!\rho\vec f\cdot\vec V+\ldots$\quad\;\scriptsize{(BE2)}&$\partial_t\{e+\frac{V^2}{2}\}+\vec V\cdot\vec\nabla(e+\frac{V^2}{2})=\vec f\!\cdot\!\vec V+\frac 1\rho\ldots$\quad\;\scriptsize{(BE4)}\\
	\hline
	\end{tabularx}\\
	\textit{Dim:} $\dot Q+\dot L_\mathrm{vol}+\dot L_\mathrm{sup}=\frac{D\mathcal{E}}{Dt}\;\;\Rightarrow\;\;
	\int_{\partial\Omega(t)}\vec q\cdot(-\hat n)\,\de (\partial\Omega)+
	\int_{\Omega(t)}\rho\vec f\cdot\vec V\,\de \Omega+
	\int_{\partial\Omega(t)}\vec t_{\hat n}\cdot\vec V\,\de (\partial\Omega)=
	D_t\int_{\Omega(t)}\rho(e+\frac{V^2}{2})\,\de \Omega$	\\
	Posso rielaborare il termine\quad$\int_{\partial\Omega(t)}\vec t_{\hat n}\cdot\vec V\,\de (\partial\Omega)=\int_{\partial\Omega(t)}\vec V\cdot\utilde T\hat n\,\de (\partial\Omega)=\int_{\partial\Omega(t)}\utilde T^\dagger\vec V\cdot\hat n\,\mathrm  d(\partial\Omega)=\int_{\partial\Omega(t)}\utilde T\vec V\cdot\hat n\,\mathrm  d(\partial\Omega)$\\
	Applico il TdD al primo membro\quad$\int_{\partial\Omega(t)}\rho\vec f\cdot\vec V+\int_{\partial\Omega(t)}\mathrm{div}(-\vec q+\utilde T\vec V)\,\de \Omega=D_t\int_{\Omega(t)}\rho(e+\frac{V^2}{2})\,\de \Omega$\\
	Per le forme conservative\quad$D_t\int_{\Omega(t)}\rho(e+\frac{V^2}{2})\,\de \Omega=
	\int_{\Omega(t)}[D_t\{\rho(e+\frac{V^2}{2})\}+\rho(e+\frac{V^2}{2})\,\mathrm{div}\,\vec V]\,\de \Omega
 	=\int_{\Omega(t)}[\partial_t\{\rho(e+\frac{V^2}{2})\}+\mathrm{div}(\rho(e+\frac{V^2}{2})\vec V)]\,\de \Omega$\\
 	Per le forme non conservative\quad$D_t\int_{\Omega(t)}\rho(e+\frac{V^2}{2})\,\de \Omega=
 	\int_{\Omega(t)}\rho D_t\{e+\frac{V^2}{2}\}\,\de \Omega
 	=\int_{\Omega(t)}\rho[\partial_t\{e+\frac{V^2}{2}\}+\vec V\cdot\vec\nabla(e+\frac{V^2}{2})]\,\de \Omega$
	
	\section{Equazioni di Navier-Stokes}	\label{ns}
	\underline{Def:}$$\begin{cases}
	D_t\rho+\rho\,\mathrm{div}\vec V=0\\
	D_t(\rho\vec V)+\rho\vec V\mathrm{div}\vec V=\rho\vec f+\mathrm{div}\,\utilde T\\
	D_t\{\rho(e+\frac{V^2}{2})\}+\rho(e+\frac{V^2}{2})\mathrm{div}\vec V=\rho\vec f+\mathrm{div}(-\vec q+\utilde T\vec V)
	\end{cases}$$
	$$\begin{cases}
	\partial_t\rho+\mathrm{div}(\rho\vec V)=0\\
	\partial_t\vec V+\vec V\cdot\utilde\nabla\vec V=\vec f+\frac 1\rho\mathrm{div}\,\utilde T\\
	\partial_t\{e+\frac{V^2}{2}\}+\vec V\cdot\vec\nabla(e+\frac{V^2}{2})=\vec f+\frac 1\rho\,\mathrm{div}(-\vec q+\utilde T\vec V)
	\end{cases}$$\\
	\textit{Dim:} prendo la forma conservativa mista e la forma non conservativa euleriana.\\
	\textit{Oss:} il sistema di Navier-Stokes è un sistema di PDE accoppiate, evolutivo, non lineare. È composto da 5 equazioni scalari indipendenti e 14 incognite: $\rho$(1), $\vec V$(3), $\utilde T$(6), $e$(1), $\vec q$(3). Il problema definito dal sistema di NS con opportune condizioni iniziale è quindi indeterminato.
		
	\section{Altre equazioni per l'energia}
		\subsection{Bilancio energia cinetica}	\label{bil en cin}
		\underline{Def:} $\frac{D(\frac{V^2}{2})}{Dt}=\rho\vec f\cdot\vec V+\vec V\cdot\mathrm{div}\,\utilde T$\quad\sz{(BC) }\\
		\textit{Dim:} partendo da (\ref{bq}.\sz{BQ3}) moltiplico scalarmente per $\vec V$ ambo i membri $\vec V\cdot\rho D_t\vec V=\rho\vec f\cdot\vec V+\vec V\cdot\mathrm{div}\,\utilde T$.\\
		\textit{Oss:} poiché è stata ottenuta da un equazione di bilancio, questa non può essere usata al posto di (\ref{be}).
		\subsection{Bilancio energia interna}
		\underline{Def:} $\rho D_te=-\mathrm{div}\,\vec q-p_e\mathrm{div}\,\vec V+\Phi$\quad\sz{(BI) }\\
		\textit{Dim:} partendo da (\ref{be}.\sz{ BE3 }) sottraggo (\ref{bil en cin}) ottenendo $\rho D_te=-\mathrm{div}\,\vec q+\big(\mathrm{div}(\utilde T\vec V\big)-\vec V\cdot\mathrm{div}\,\utilde T)=-\mathrm{div}\,\vec q+\utilde T\cdot\utilde\nabla\vec V$. La quantità $\utilde T\cdot\utilde\nabla\vec V$ può essere convenientemente riscritta decomponendo $\utilde T$ nella forma (iso)+(dev). Per un continuo fluido \big(vd. (\ref{fluido})\big) $\utilde T=-p_e\utilde I+\utilde\tau$ quindi $\utilde T\cdot\utilde\nabla\vec V=-p_e\utilde I\cdot\utilde\nabla\vec V+\utilde\tau\cdot\utilde\nabla\vec V=-p_e\,\mathrm{div}\,\vec V+\Phi$.\\
		\textit{Oss:} la quantità $-p_e\,\mathrm{div}\,\vec V$ rappresenta la potenza (per unità di volume) delle forze di pressione, esplicata con la parte isotropa della deformazione \big(vd. (\ref{def elem})\big). La funzione $\Phi=\utilde\tau\cdot\utilde\nabla\vec V$ è chiamata funzione di dissipazione e rappresenta la potenza (per unità di volume) delle forze viscose, esplicata con la parte non isotropa della deformazione.

%%%%%%%%%%%%%%%%%%%%%%%%%%%%%%%%%%%%%%%%%%%%%%%%%%%%%%%%%%%%%%%%%%%%%%%%%%%%%%%%%%%%%%%%%%%%%%%%%%%%%%%%%%%%%%%%%%%%%%%%%%%%%%%%%%%%%%%%%%%%%%%%%%%%%%%%%%%%%%%%%%%%%%%%%%%%%%%%%%%%%%%%%%%%%%%%%%%%%%%%%%%%%%%%%%%%%%%%%%%%%%%%%%%%%%%%%%%%%%%%%%%%%%%%%%%%%%%%%%%%%%%%%%%%%%%%%%%%%%%%%%%%%%%%%%%%%%%%%%%%%%%%%%%%%%%%%%%%%%%%%%%%%%%%%%%%%%%%%%%%%%%%%%%%%%%%%%%%%%%%%%%%%%%%%%%%%%%%%%%%%%%%%%%%%%%%%%%%%%%%%%%%%%%%%%%%%%%%%%%%%%%%%%%%%%%%%%%%%%%%%%%%%%%%%%%%%%%%%%%%%%
 
\chapter{Modello di continuo fluido}
%\section{Vorticità e flusso irrotazionale} \label{pot vel}
%	\underline{Def:} campo vettoriale $\vec\omega(P,t)=\mathrm{rot}\,\vec V(P,t)$. Se $\vec\omega(P,t)\equiv\vec 0$ il flusso si dice irrotazionale e si può scrivere $\vec V=\vec\nabla\phi$ dove $\phi(P,t)$ è il potenziale di velocità\\
%	\textit{Dim:} il potenziale di velocità segue dal fatto che se $\mathrm{rot}\,\vec f=0$ allora $\vec f$ ammette un potenziale scalare
	\section{Continuo fluido} \label{fluido}
	\underline{Def:} sperimentalmente si trova che un continuo fluido ha la proprietà, in condizioni di equilibrio, di presentare forze di superficie del tipo $\vec t_{\hat n}=t_{\hat n}\,\hat n$ e $t_{\hat n}\leq 0\quad\forall P\in\mathcal C_t\;\;\forall \hat n\in S^2$\\
	\textit{Oss:} per un continuo fluido in equilibrio ho $\vec V=0$ e $\mathcal C_t=\mathcal C_0$ quindi nelle grandezze non c'è dipendenza dal tempo
	
	\section{Volume cubico di fluido}
	\underline{Def:} cubo definito tramite due particelle materiali $A$ e $B$ (molto vicine), che costituiscono gli estremi di una sua diagonale
	
	\section{Tensioni interne e tensore delle tensioni}	\label{tensioni fluido}
	\underline{Def:} $\vec t_{\hat n}=-p_\mathrm e \hat n+\vec\tau_{\hat n}\qquad
	\utilde T=-p_\mathrm e\utilde I+\utilde\tau\;\Rightarrow\;\mathrm{div}\,\utilde T=-\vec\nabla p+\mathrm{div}\,\utilde\tau$\\
	\textit{Dim:} tenendo conto di (\ref{fluido}) e anche della decomposizione (iso)+(dev) di un tensore. Per il teorema di Cauchy\\$t_{\hat n}=\utilde T\,\hat n=(-p_\mathrm e\utilde I+\utilde\tau)\,\hat n=-p_\mathrm e\utilde I\,\hat n+\utilde\tau\,\hat n=-p_\mathrm e\hat n+\vec\tau_{\hat n}$. Applicando l'operatore divergenza $\mathrm{div}\,\utilde T=\mathrm{div}(-p_\mathrm e\utilde I+\utilde\tau)=\mathrm{div}(-p_\mathrm e\utilde I)+\mathrm{div}\,\utilde\tau=-\vec\nabla p_\mathrm e+\mathrm{div}\,\utilde\tau$\\
	\textit{Oss:}  le tensioni agenti su un elemento $\de \Omega$ di fluido possono essere distinte tra tensioni dovute alla pressione e tensioni dovuti agli sforzi tangenziali. Per un fluido all'equilibrio questi ultimi sono assenti quindi $\utilde T=-p_\mathrm e\utilde I$.
	
	\section{Forze aerodinamiche}
	\underline{Def:} 
	\textit{Dim:} dato un continuo che si muove in un fluido, la forza agente su una superficie elementare della frontiera è $\de \vec F=-p\hat n\,\de S+\vec\tau_{\hat n}\,\de S$ quindi la risultante è $\vec F=-\int_{\partial\mathcal C}p\hat n\,\de S+\int_{\partial\mathcal C}\vec\tau_{\hat n}\,\de S$. La resistenza aerodinamica è la componente in direzione del flusso asintotico $D=\vec F\cdot\mathrm{vers}(\vec U_\infty)$; la portanza è la componente  in direzione ortogonale a quest'ultima $L=\vec F\cdot\mathrm{vers}(\vec U_\infty^\perp)$. Questo sistema di forze è equivalente al sistema $\{(C,\vec F)+\vec M^{(\mathrm C)}\}$ in cui $C$ è il centro aerodinamico del corpo.

	\section{Equazioni di Navier-Stokes per un fluido generico} \label{ns fluido}
	\underline{Def:}$$\begin{cases}
	D_t\rho+\rho\,\mathrm{div}\vec V=0\\
	D_t(\rho\vec V)+\rho\vec V\mathrm{div}\vec V=\rho\vec f+-\vec\nabla p_\mathrm e+\mathrm{div}\,\utilde\tau\\
	D_t\{\rho(e+\frac{V^2}{2})\}+\rho(e+\frac{V^2}{2})\mathrm{div}\,\vec V=\rho\vec f\cdot\vec V+\mathrm{div}(-\vec q-p_\mathrm e\vec V+\utilde\tau\vec V)
	\end{cases}$$
	$$\begin{cases}
	\partial_t\rho+\mathrm{div}(\rho\vec V)=0\\
	\partial_t\vec V+\vec V\cdot\utilde\nabla\vec V=\vec f+-\frac 1\rho\vec\nabla p_\mathrm e+\frac 1\rho\mathrm{div}\,\utilde\tau\\
	\partial_t\{e+\frac{V^2}{2}\}+\vec V\cdot\vec\nabla(e+\frac{V^2}{2})=\vec f\cdot\vec V+\frac 1\rho\,\mathrm{div}(-\vec q-p_\mathrm e\vec V+\utilde\tau\vec V)
	\end{cases}$$\\
    \textit{Oss:} avendo introdotto la pressione termodinamica nelle (\ref{ns}) ho sempre 5 equazioni scalari indipendenti mentre il numero di incognite è ?? $\rho$(1), $\vec V$(3), $p_\mathrm e$(1), $\utilde\tau$(?), $e$(1), $\vec q$(3). Il problema è ancora indeterminato.
    
	\section{Velocità del suono e numero di Mach}
	\underline{Def:} \\
	\textit{Dim:} considero un sistema pistone-cilindro riempito con del fluido comprimibile. Il pistone compie un'oscillazione con velocità $\vec v=a\,\hat\imath$, generando un disturbo di pressione che si propaga nel fluido con velocità $a$. Mi metto solidale all'onda. A valle di essa vedo il fluido muoversi verso l'onda con velocità $\vec V=-a\,\hat\imath$ e proprietà indisturbate $p,\rho,T$. A monte dell'onda vedo il fluido allontanarsi da essa con velocità $\vec V=-(a-\delta u)\,\hat\imath$ e proprietà perturbate $p+\delta p,\rho+\delta\rho,T+\delta T$. Prendo un VC a ridosso dell'onda e applico (\ref{ns fluido}) in forma convettiva. Per \sz{(BM)  } trovo $\rho a=(\rho+\delta\rho)(a-\delta u)$ cioè $\delta u\simeq a\frac{\delta\rho}\rho$. Per \sz{(BQ)  } trascurando sforzi tangenziali e forze di massa
    
    \section*{Modello dell'atmosfera standard}
	composizione chimica costante: 78\% $N_2$, 21\% $O_2$, 1\% altri gas\\
	$\de p=-\rho g(z)\de z$\\
	$g(z)=g_0(\frac{R_0}{z+R_0})^2\qquad g_0=9,807\;\mathrm{ms^{-2}}\quad R_0=6356\,\text{ km}$\\
	$p=\rho R_\mathrm{aria}T\qquad R_\mathrm{aria}=287.1\;\mathrm{\frac{m^2}{s^2\,K}}$\\
	$\de T=-0.0065\mathrm{[\frac{K}{M}]}\!\cdot\!\de z$ per $0<z<11000$, $T=\mathrm{cost}$ per $11000<z<20000$\\
	$\rho(0)=1.225\mathrm{[\frac{kg}{m^3}]},p(0)=1\mathrm{[atm]}=101325\mathrm{[Pa]},T(0)=15\mathrm{[C]}=288.15\mathrm{[K]},g(0)=g_0$
	
	\section*{Valori da ricordare}
	$a_\mathrm{aria}=340,3\;\mathrm{\frac ms}\qquad h=0\,\text{m}\quad T=15^\circ\text{C}\quad p=1\,\text{atm}$\\
	$\rho_\mathrm{aria}=1,225\;\mathrm{\frac{kg}{m^3}}\qquad h=0\,\text{m}\quad T=15^\circ\text{C}\quad p=1\,\text{atm}$\\
	$\mu_\mathrm{aria}=1,78\cdot 10^{-5}\;\mathrm{\frac{kg}{m\:s}}$\\
	$R_\mathrm{aria}=287,1\;\mathrm{\frac{J}{kg\:K}}$\\
	$\gamma_\mathrm{aria}=1,4$


%%%%%%%%%%%%%%%%%%%%%%%%%%%%%%%%%%%%%%%%%%%%%%%%%%%%%%%%%%%%%%%%%%%%%%%%%%%%%%%%%%%%%%%%%%%%%%%%%%%%%%%%%%%%%%%%%%%%%%%%%%%%%%%%%%%%%%%%%%%%%%%%%%%%%%%%%%%%%%%%%%%%%%%%%%%%%%%%%%%%%%%%%%%%%%%%%%%%%%%%%%%%%%%%%%%%%%%%%%%%%%%%%%%%%%%%%%%%%%%%%%%%%%%%%%%%%%%%%%%%%%%%%%%%%%%%%%%%%%%%%%%%%%%%%%%%%%%%%%%%%%%%%%%%%%%%%%%%%%%%%%%%%%%%%%%%%%%%%%%%%%%%%%%%%%%%%%%%%%%%%%%%%%%%%%%%%%%%%%%%%%%%%%%%%%%%%%%%%%%%%%%%%%%%%%%%%%%%%%%%%%%%%%%%%%%%%%%%%%%%%%%%%%%%%%%%%%%%%%%%%%

\chapter{Fluidi comprimibili}
	\section{Variazione di configurazione}
	\underline{Def:} $\Delta\vec x=(\utilde\nabla\vec V\,\delta\vec x)\,\de t\qquad \Delta\upsilon=(\delta x+\Delta x)(\delta y+\Delta y)(\delta z+\Delta z)-\delta x\delta y\delta z$\\
	\textit{Dim:} prendo un volume cubico, all'istante $t$, definito da $A(t)\equiv(x,y,z)$ e $B(t)\equiv(x\!+\!\delta x,y\!+\!\delta y,z\!+\!\delta z)$ di velocità $\vec V_A=(u(A),v(A),w(A))\equiv(u,v,w)$ e $\vec V_B=(u(B),v(B),w(B))\equiv(u(A)\!+\!\delta u,v(A)\!+\!\delta v,w(A)\!+\!\delta w)\neq\vec V_A$.
	La diagonale è $\overrightarrow{AB}(t)\equiv \delta \vec x=(\delta x,\delta y,\delta z)$ e dopo un intervallo $\de t$ diventa $\overrightarrow{AB}(t\!+\!\de t)\equiv\delta \vec x\!+\!\Delta\vec x=(\delta x\!+\!\Delta x,\delta y\!+\!\Delta y,\delta z\!+\!\Delta z)$. Le variazioni di coordinate $\Delta x_k=\delta u_k\de t$ sono dovute alla differenza delle velocità delle particelle. Sviluppando secondo Taylor al primo ordine ho $\Delta x_k=\frac{\partial u_k}{\partial x_i}\delta x^i\,\de t$. Se definisco il tensore gradiente di velocità $[\utilde\nabla\vec V]_{ij}=(\partial_ju_i)$, posso pensare di ottenere la variazione di posizione relativa come risultato della sua applicazione alla posizione relativa iniziale $\Delta \vec x=\utilde\nabla\vec V\,\delta\vec x$\\
	\textit{Oss:} a meno di una traslazione rigida di entità $\vec V\,\de t$ la nuova configurazione del volume cubico è caratterizzata dalla variazione di posizione relativa delle particelle, che è determinata dal tensore gradiente di velocità. Nell'ambito della meccanica dei fluidi si preferisce utilizzare una descrizione euleriana dei fenomeni poiché intervengono effetti dinamici che causano il moto del continuo fluido, quindi è innaturale assumere una CR rispetto alla quale studiarne gli effetti
	
	\section{Gradiente di velocità}	\label{grad vel}
	\underline{Def:} $\utilde\nabla\vec V=\utilde E+\utilde \Omega=\utilde A+\utilde B+\utilde C+\utilde \Omega\qquad \utilde E\in Sym(3),\,\utilde \Omega\in Skw(3)\quad \utilde A,\utilde B\in Diag(3)$\\$
	[\utilde A]=\left[\begin{matrix}
	\frac 13\mathrm{div}\vec V&0&0\\
	0&\frac 13\mathrm{div}\vec V&0\\
	0&0&\frac 13\mathrm{div}\vec V
	\end{matrix}\right]$\qquad$
	[\utilde B]=\left[\begin{matrix}
	\frac{\partial u}{\partial x}-\frac 13\mathrm{div}\vec V&0&0\\
	0&\frac{\partial v}{\partial y}-\frac 13\mathrm{div}\vec V&0\\
	0&0&\frac{\partial w}{\partial z}-\frac 13\mathrm{div}\vec V
	\end{matrix}\right]$\\$
	[\utilde C]=\left[\begin{matrix}
	0&\frac 12(\frac{\partial v}{\partial x}+\frac{\partial u}{\partial y})&\frac 12(\frac{\partial w}{\partial x}+\frac{\partial u}{\partial z})\\
	\frac 12(\frac{\partial u}{\partial y}+\frac{\partial v}{\partial x})&0&\frac 12(\frac{\partial w}{\partial y}+\frac{\partial v}{\partial z})\\
	\frac 12(\frac{\partial u}{\partial z}+\frac{\partial w}{\partial x})&\frac 12(\frac{\partial v}{\partial z}+\frac{\partial w}{\partial x})&0
	\end{matrix}\right]$\qquad$
	[\utilde \Omega]=\left[\begin{matrix}
	0&\frac 12(\frac{\partial v}{\partial x}-\frac{\partial u}{\partial y})&\frac 12(\frac{\partial w}{\partial x}-\frac{\partial u}{\partial z})\\
	\frac 12(\frac{\partial u}{\partial y}-\frac{\partial v}{\partial x})&0&\frac 12(\frac{\partial w}{\partial y}-\frac{\partial v}{\partial z})\\
	\frac 12(\frac{\partial u}{\partial z}-\frac{\partial w}{\partial x})&\frac 12(\frac{\partial v}{\partial z}-\frac{\partial w}{\partial y})&0
	\end{matrix}\right]$\\
	\textit{Dim:} per definizione è $\utilde A:=\frac 13 \mathrm{tr}(\utilde\nabla\vec V)\utilde I=\frac 13\mathrm{div}\vec V\utilde I$ quindi deve essere $[\utilde B]_{ij}=(\partial_iu_j-\frac 13\mathrm{div}\vec V)\delta_{ij}$. Gli elementi extra diagonali li spartisco fra $[\utilde C]$ e $[\utilde \Omega]$ in modo tale che $[\utilde \Omega]_{ij}:=\frac 12(\partial_iu_j-\partial_ju_i)$ quindi deve essere $[\utilde C]_{ij}=\frac 12(\partial_iu_j+\partial_ju_i)-\partial_ju_i\delta_{ij}$.\\
	\textit{Oss:} osservo che $[\utilde B+\utilde C]_{ij}=\frac 12(\partial_iu_j+\partial_ju_i)-\frac 13\mathrm{div}\delta_{ij}\,\vec V$\quad\big(vd. (\ref{hp newton})\big).
	
	\subsection{Deformazioni elementari}	\label{def elem}
	\underline{Def:} i contributi di $\utilde A,\utilde B$ e $\utilde C$ alla deformazione del volume cubico di fluido sono elementari. Dalla loro composizione posso ottenere qualsiasi tipo di deformazione\\
	$\utilde A\,\delta\vec x=\frac 13\mathrm{div}\vec V(\delta x,\delta y,\delta z)$\quad deformazione isotropa\\
	$\utilde B\,\delta\vec x=((\frac{\partial u}{\partial x}\!-\!\frac13\mathrm{div}\vec V)\delta x,(\frac{\partial v}{\partial y}\!-\!\frac13\mathrm{div}\vec V)\delta y,(\frac{\partial w}{\partial z}\!-\!\frac13\mathrm{div}\vec V)\delta z)$\quad deformazione lineare isovolumica\\
	$\utilde C\,\delta\vec x=\frac 12((\frac{\partial v}{\partial x}\!+\!\frac{\partial u}{\partial y})\delta y\!+\!(\frac{\partial w}{\partial x}\!+\!\frac{\partial u}{\partial z})\delta z,(\frac{\partial u}{\partial y}\!+\!\frac{\partial v}{\partial x})\delta x\!+\!(\frac{\partial w}{\partial y}\!+\!\frac{\partial v}{\partial z})\delta z,(\frac{\partial u}{\partial y}\!+\!\frac{\partial v}{\partial x})\delta x\!+\!(\frac{\partial w}{\partial y}\!+\!\frac{\partial v}{\partial z})\delta z)$\quad distorsione pura\\
	$\utilde\Omega\,\delta\vec x= ???\overset{(1.1.5)}{=}\frac 12\vec\omega\times\delta\vec x$\quad rotazione rigida\\
	\textit{Oss:} si dimostra che le variazioni di volume corrispondenti sono:\\
	$(\Delta\upsilon)_{\utilde A}\cong\upsilon(t)\mathrm{div}\vec V\,\de t$\\
	$(\Delta\upsilon)_{\utilde B}\cong\mathrm{tr}\utilde B\,\mathrm{div}\vec V\,\de t=0$\\
	$(\Delta\upsilon)_{\utilde C}\cong 0$\\
	$(\Delta\upsilon)_{\utilde \Omega} ??$
	
	\section{Equazioni costitutive}	\label{equaz cost}
		\subsection{Equazione termica di stato}
		\underline{Def:} $p=p(\Theta,\rho)$\\
		\textit{Oss:} per un gas termicamente perfetto $p=\rho R\Theta$
		\subsection{Equazione calorica di stato}
		\underline{Def:} $e=e(\Theta,\rho)$ oppure $h=h(\Theta,\rho)$. Per un gas termicamente perfetto $e=e(\Theta)$ oppure $h=h(\Theta)$\\
		\textit{Oss:} per un gas caloricamente perfetto $e=c_v\Theta$ oppure $h=c_p\Theta$, con $c_v$ e $c_p$ costanti
		\subsection{Equazione della trasmissione del calore}
		\underline{Def:} $\vec q=-\utilde\kappa(\Theta)\vec\nabla\Theta$\qquad se il continuo è isotropo $\vec q=-k(\Theta)\vec\nabla\Theta$\\
		\textit{Oss:} per un gas perfetto $k\propto\sqrt{\Theta}$. Questa legge descrive solo lo scambio di calore per conduzione; se voglio includere altri meccanismi (irraggiamento, reazioni chimiche, ...) devo aggiungere i corrispondenti termini nell'equazione di bilancio dell'energia
		\subsection{Proprietà intrinseche}
		\underline{Def:} valgono le ipotesi dovute a Stokes $(i)$ omogeneità del fluido\quad$(ii)$ isotropia del fluido\quad$(iii)$\phantom{ }$\utilde\tau=f(\utilde E)$\\$(iv)$ se $\vec V=\vec 0\:(\Leftrightarrow\utilde E=\utilde 0)$ allora $\utilde\tau=\utilde 0$\\
		\textit{Oss:} non si è specificata la forma del legame funzionale tra $\utilde\tau$ e $\utilde E$
		
	\section{Ipotesi di Newton}	\label{hp newton}
	\underline{Def:} $\utilde\tau=2\mu\utilde E+\lambda\,\mathrm{div}\,\vec V\utilde I=2\mu(\utilde B+\utilde C)+\eta\,\mathrm{div}\,\vec V\utilde I\qquad\qquad\tau_{ij}=\mu(\partial_iu_j+\partial_ju_i)+\lambda\,\mathrm{div}\,\vec V\delta_{ij}\\\mu$ - coeff. di viscosità molcolare\quad$\lambda$ - coeff. di stato termodinamico\quad$\eta=\lambda+\frac 23\mu$ - coeff. di viscosità di dilatazione\\
	\textit{Dim:} $2\mu\utilde E+\lambda\,\mathrm{div}\,\vec V\utilde I\overset{(\ref{grad vel})}{=}2\mu(\frac 13\mathrm{div}\,\vec V\utilde I+\utilde B+\utilde C)+\lambda\,\mathrm{div}\,\vec V\utilde I=2\mu(\utilde B+\utilde C)+(\lambda+\frac 23\mu)\mathrm{div}\,\vec V\utilde I$.\\Le componenti seguono tenendo conto dell'osservazione (\ref{grad vel}).\\
	\textit{Oss:} per i fluidi newtoniani il legame funzionale tra $\utilde\tau$ e $\utilde E$ è lineare.\\
	\phantom{\qquad}\fbox{Da ora in poi si ritiene implicitamente valida l'ipotesi di Newton}
		\subsection{Tensore degli sforzi con hp. di Newton}	\label{tensore newton}
		\underline{Def:} $\utilde T=(-p_e+\eta\,\mathrm{div}\,\vec V)\utilde I+2\mu(\utilde B+\utilde C)\qquad
	T_{ij}=(-p_e+\eta\,\mathrm{div}\,\vec V)\delta_{ij}+\mu(\partial_iu_j+\partial_ju_i)-\frac 23 \mu\,\mathrm{div}\,\vec V\delta_{ij}$\\
		\textit{Dim:} $\utilde T\overset{(\ref{tensioni fluido})}{=}-p_e\,\utilde I+\utilde\tau\overset{(\ref{hp newton})}{=}-p_e\,\utilde I+\eta\,\mathrm{div}\,\vec V\utilde I+2\mu(\utilde B+\utilde C)=(-p_e+\eta\,\mathrm{div}\,\vec V)\utilde I+2\mu(\utilde B+\utilde C)$.\\Analogamente le componenti $T_{ij}=(-p_\mathrm e+\eta\,\mathrm{div}\,\vec V)\delta_{ij}+\mu(\partial_iu_j+\partial_ju_i)-\frac 23\mu\,\mathrm{div}\,\vec V\delta_{ij}=(-p_\mathrm e+\lambda\,\mathrm{div}\,\vec V)\delta_{ij}+\mu(\partial_iu_j+\partial_ju_i)$.\\
		\textit{Oss:} il tensore così scritto presenta una parte isotropa e una deviatorica: per il generico elemento $T_{ij}^{(\mathrm{iso})}=(-p_e+\eta\,\mathrm{div}\,\vec V)\delta_{ij}$ e $T_{ij}^{(\mathrm{dev})}=\mu(\partial_iu_j+\partial_ju_i)-\frac 23 \mu\,\mathrm{div}\,\vec V\delta_{ij}$.
	
	\section{Ipotesi di Stokes e pressione meccanica} \label{hp stokes}
	\underline{Def:} $\eta\,\mathrm{div}\,\vec V\approx 0\qquad\qquad p=\frac 13 \sum_i T_{ii}=p_\mathrm e-\eta\,\mathrm{div}\vec V$\\
	\textit{Dim:} considero la decomposizione $\utilde T=-p\utilde I+\utilde{\mathcal T}$ in cui $[\utilde{\mathcal T}]=\left[\begin{matrix}T_{11}-p&T_{12}&T_{13}\\T_{21}&T_{22}-p&T_{23}\\T_{31}&T_{32}&T_{33}-p\end{matrix}\right]$. Anche qui definisco (per costruzione) la parte isotropa $T_{ij}^{(\mathrm{iso})}=-p\,\delta_{ij}$ e la parte deviatorica $T_{ij}^{(\mathrm{dev})}=[\utilde{\mathcal T}]_{ij}$. Uguagliando queste alle (\ref{tensore newton}) ottengo $-p=-p_\mathrm e+\eta\,\mathrm{div}\,\vec V\quad\mathcal T_{ij}=\mu(\partial_iu_j+\partial_ju_i)-\frac 23 \mu\,\mathrm{div}\,\vec V\delta_{ij}$ da cui la relazione tra pressione meccanica e termodinamica.\\
	\textit{Oss:} sotto tale ipotesi il pedice\, "e"\, perde di significato quindi viene omesso. [DISCUTI VALIDITA DELL IP]\\
	\phantom{\qquad}\fbox{Da ora in poi si ritiene implicitamente valida l'ipotesi di Stokes}
		\subsection{Tensore degli sforzi con hp. di Stokes}
		\underline{Def:} $\utilde T=-p\utilde I+2\mu(\utilde B+\utilde C)\qquad	T_{ij}=-p_e\delta_{ij}+\mu(\partial_iu_j+\partial_ju_i)-\frac 23 \mu\,\mathrm{div}\,\vec V\delta_{ij}$\\
		\textit{Dim:} segue da (\ref{hp newton}) tenendo conto di (\ref{hp stokes}). Sostanzialmento l'ip. di Stokes permette di semplificare la scrittura degli elementi $\tau_{ij}$
			
	\section{Sistema di Navier-Stokes per un gas perfetto} \label{nsgp}
	\underline{Def:}$$\begin{cases}
	D_t\rho+\rho\,\mathrm{div}\vec V=0\\
	D_t(\rho\vec V)+\rho\vec V\mathrm{div}\vec V=\rho\vec f-\vec\nabla p+\mathrm{div}\,\utilde\tau\\
	D_t\{\rho(e+\frac{V^2}{2})\}+\rho(e+\frac{V^2}{2})\mathrm{div}\vec V=\rho\vec f\cdot\vec V+\mathrm{div}(-\vec q-p\vec V+\utilde\tau\vec V)\\
	\vec q=-k\vec\nabla\Theta\\
	p=\rho R \Theta\qquad e=c_v\Theta\\
	T_{ij}=-p\,\delta_{ij}+\mu(\partial_iu_j+\partial_ju_i)-\frac 23\mu\mathrm{div}\vec V\\
	\end{cases}$$
	\textit{Dim:} dalle (\ref{nsf}), aggiungendo le 3 componenti del postulato di Fourier, le 2 equazioni di stato e le 6 equazioni costitutive, ottengo un sistema di 16 equazioni nelle 16 incognite: $\rho$(1), $\vec V$(3), $p$(1), $\utilde\tau$(6), $e$(1), $\vec q$(3), $\Theta$(1). Il sistema è quindi virtualmente risolvibile [in forma chiusa?]. Solitamente si assumono $\rho$, $\vec V$ ed $e$ (oppure $\Theta$) come incognite principali e si determinano le restanti mediante le relazioni costitutive.
	
	\section{Condizioni al contorno per parete solida}	\label{cond contorno}
	\underline{Def:} condizioni sulla velocità: $\begin{cases}\vec V(\text{parete},t)=\vec V_\text{parete}&$cond. di aderenza$\\\vec V\cdot\hat n=0&$cond. di non penetrabilità$\end{cases}$\\
	condizioni sulla temperatura: $\begin{cases}T(\mathrm{parete},t)=T_\mathrm{parete}=\mathrm{cost}&$parete isoterma$\\\vec q(\mathrm{parete},t)=\vec 0&$parete adiabatica$\end{cases}$\\
	\textit{Oss:} $\vec q(\mathrm{parete})=\vec 0$ significa che non c'è flusso di calore in direzione perpendicolare alla parete, $-k\vec\nabla T\cdot\hat n_\mathrm{parete}=0\;\Rightarrow\;\vec\nabla T\cdot\hat n_\mathrm{p}=0\;\Rightarrow\;\frac{\partial T}{\partial\hat n}|_{\mathrm{parete}}=0$		
	
	\section{Risoluzione analitica e numerica}
	\underline{Def:} soluzioni analitiche esatte delle equazioni di NS per un gas perfetto esistono in pochi casi lontani dall'interesse aeronautico. La simulazione numerica è possibile, anche per configurazioni realistiche, ma con tempi di calcolo lunghi e impraticabili dal punto di vista progettuale\\
	\textit{Oss:} i tempi tipici di simulazione sono dell'ordine delle settimane/mesi. Poichè in fase di progetto è necessario avere una risposta immediata, queste applicazioni trovano impiego solo in fase di verifica e collaudo. Da qui nasce la necessità di introdurre ulteriori ipotesi semplificative.
		
%%%%%%%%%%%%%%%%%%%%%%%%%%%%%%%%%%%%%%%%%%%%%%%%%%%%%%%%%%%%%%%%%%%%%%%%%%%%%%%%%%%%%%%%%%%%%%%%%%%%%%%%%%%%%%%%%%%%%%%%%%%%%%%%%%%%%%%%%%%%%%%%%%%%%%%%%%%%%%%%%%%%%%%%%%%%%%%%%%%%%%%%%%%%%%%%%%%%%%%%%%%%%%%%%%%%%%%%%%%%%%%%%%%%%%%%%%%%%%%%%%%%%%%%%%%%%%%%%%%%%%%%%%%%%%%%%%%%%%%%%%%%%%%%%%%%%%%%%%%%%%%%%%%%%%%%%%%%%%%%%%%%%%%%%%%%%%%%%%%%%%%%%%%%%%%%%%%%%%%%%%%%%%%%%%%%%%%%%%%%%%%%%%%%%%%%%%%%%%%%%%%%%%%%%%%%%%%%%%%%%%%%%%%%%%%%%%%%%%%%%%%%%%%%%%%%%%%%%%%%%%
	
\chapter{Fluidi incomprimibili con flusso irrotazionale}
	\section{Fluido incomprimibile} \label{def incomp}
	\underline{Def:} $\rho(P,t)=\rho_0\quad\forall P\in B\quad\forall t$, quindi $\de \rho=\partial_i\rho\,\de x_i+\partial_t\rho\,\de t=0$ e $D_t\rho=0$. Inoltre se $\mathrm{Ma}<0.3$ l'ipotesi di flusso incomprimibile è giustificata\\
	\textit{Dim:} dimostrazione che $\frac{\Delta\rho}{\rho}\sim\mathrm{Ma}^2$\\
	\textit{Oss:} il campo della densità è uniforme e costante.
		\subsection{Tensore degli sforzi con hp. di fluido incomprimibile}	\label{tens incomp}
		\underline{Def:} $T_{ij}=-p\delta_{ij}+\mu(\partial_i u_j+\partial_j u_i)\qquad\mathrm{div}\,\utilde\tau=\mu\nabla^2\vec V$\\
		\textit{Dim:} segue dalle (\ref{tensore newton}) semplificando $\mathrm{div}\,\vec V=0$ grazie a (\ref{ns inc}). Per la divergenza ottengo $(\mathrm{div}\,\utilde\tau)_i=\mu\big(\mathrm{div}(\utilde B+\utilde C)\big)_i=\mu\,\partial_j[\utilde B+\utilde C]_{ji}=\mu\,\partial_j(\partial_ju_i+\partial_iu_j)=\mu\,(\partial_{jj}u_i+\partial_{ij}u_j)=\mu(\nabla^2u_i+\partial_i\,\mathrm{div}\,\vec V)\overset{(\ref{def incomp})}{=}\mu\nabla^2 u_i.$\\
		\textit{Oss:} la notazione di Newton sottointende la sommatoria sugli indici ripetuti.
		
	\section{Ipotesi di conservatività}
	\underline{Def:} $\vec f=-\vec\nabla\psi\quad\forall\vec f$ campo di forze di massa\\
	\textit{Dim:} l'assunzione è giustificata dal fatto che, oltre ad aver supposto il continuo esente da cariche, in ambito aeronautico l'unico campo di forze presente nella realtà fisica è quello gravitazionale, quindi conservativo. Inoltre, anche in presenza di un campo elettromagnetico, l'ipotesi di conservatività resta valida.\\\mbox{}\qquad
	\fbox{Da ora in poi si ritiene implicitamente valida l'ipotesi di conservatività}
	
	\section{Equazioni di Navier-Stokes per fluido incomprimibile} \label{ns inc}
	\underline{Def:}$$\begin{cases}\mathrm{div}\,\vec V=0\;\Rightarrow\;\nabla^2\phi=0\\
	D_t\vec V=-\vec\nabla\psi-\frac{1}{\rho}\vec\nabla p+\nu\nabla^2\vec V\\
	\ldots energia\ldots\end{cases}$$\\
	\textit{Dim:} parto dalle (\ref{nsgp}) con l'ipotesi di conservatività. Per \sz{ (BM) }, grazie a (\ref{def incomp}), ottengo $\mathrm{div}\,\vec V=0$.\\
	Per \sz{ (BQ) }, grazie anche a (\ref{tens incomp}), ottengo $\rho D_t\vec V=\rho\vec f-\vec\nabla p+\mu\nabla^2\vec V$. Dividendo ambo i membri per $\rho$ definisco il coeff. di viscosità cinematica $\nu=\frac{\mu}{\rho}$, da cui $D_t\vec V=-\vec\nabla\psi-\frac 1\rho\vec\nabla p+\nu\nabla^2\vec V$.\\
	Per \sz{ (BE) } \ldots\\
	\textit{Oss:} avendo introdotto l'ipotesi di fluido incomprimibile ho sempre 5 equazioni scalari indipendenti ma l'ultima è disaccoppiata dalle restanti. Queste posso risolverle per trovare le incognite scalari $u(P,t),v(P,t),w(P,t),p(P,t)$, ed eventualmente utilizzarle nell'ultima per trovare $\Theta(P,t)$, sebbene le variazioni di temperatura sono spesso trascurabili. Tuttavia anche nel caso di fluido incomprimibile la risoluzione analitica e numerica hanno gli stessi limiti di (\ref{nsgp}).
		\subsection{Bilancio della q.d.m. con vorticità}	\label{qdm vort}
		\underline{Def:} $\partial_t\vec V+\vec\omega\times\vec V+\nu\,\mathrm{rot}\,\vec\omega=\vec\nabla(\psi+\frac p\rho+\frac{V^2}{2})$\\
		\textit{Dim:} partendo da (\ref{ns inc}.\sz{ BQ }) in forma euleriana, sfrutto le due identità vettoriali $\vec V\cdot\utilde\nabla\vec V=\vec\omega\times\vec V+\vec\nabla(\frac{V^2}{2})\quad\nabla^2\vec V=-\mathrm{rot}\,\vec\omega$ per ottenere $\partial_t\vec V+\vec\omega\times\vec V+\vec\nabla(\frac{V^2}{2})=-\vec\nabla\psi-\frac 1\rho\vec\nabla p-\nu\,\mathrm{rot}\,\vec\omega$.
	
	\section{Ipotesi di irrotazionalità}
	\underline{Def:} $\mathrm{rot}\,\vec V=\vec 0$
	\begin{center}\underline{\smash{Da ora in poi si ritiene implicitamente valida l'ipotesi di irrotazionalità}}\end{center}
	
	\section{Equazioni di Navier-Stokes per flusso potenziale}	\label{ns pot}
	\underline{Def:}$$\begin{cases}\nabla^2\phi=0\\
	\partial_t\varphi+\frac{p}{\rho}+\frac{V^2}{2}+\psi=f(t)\\
	energia\end{cases}$$\\
	\textit{Dim:} parto dalla (\ref{ns inc}). Per \sz{ (BM)  } ottengo $\mathrm{div}\,\vec V=0\Rightarrow\exists\,\phi$ t.c. $\vec\nabla\phi=\vec V$, quindi $\mathrm{div}(\vec\nabla\phi)=\nabla^2\phi=0$.\\
	Per \sz{ (BQ)  } dalla (\ref{qdm vort}) introduco l'ipotesi di flusso irrotazionale per ottenere $\partial_t\{\vec\nabla\phi\}=-\vec\nabla(\frac{p}{\rho}+\frac{V^2}{2}+\psi)$. Tenendo conto che $\partial_t\{\vec\nabla\varphi\}=\vec\nabla(\partial_t\varphi)$, ottengo $\vec\nabla(\partial_t\varphi+\psi+\frac{p}{\rho}+\frac{V^2}{2})=0$. Poiché il gradiente è un operatore spaziale, questa relazione equivale a $\partial_t\varphi+\psi+\frac{p}{\rho}+\frac{V^2}{2}=f(t)$.\\
	\textit{Oss:} in un flusso irrotazionale il termine $\nu\nabla^2\vec V=\nu\,\mathrm{rot}\,\vec\omega$ dovuto agli sforzi viscosi si annulla e il fluido si comporta come se fosse non viscoso. Questo non significa che gli sforzi viscosi sono nulli, infatti nell'equazione dell'energia il termine di dissipazione in genere è non nullo; è nulla invece la risultante degli sfrozi viscosi (agenti su un VM). Per flusso potenziale si intende il flusso irrotazionale di un fluido incomprimibile, cioè un flusso per cui esiste il potenziale di velocita.
	
	\section{Ipotesi di stazionarietà}
	\underline{Def:} tutte le grandezze (e relazioni fra di esse) sono indipendenti dal tempo, in particolare $\partial_t(\cdot)\equiv 0$
	
	\section{Trinomio e teorema di Bernoulli}
	\underline{Def:} $B=\frac{p}{\rho}+\frac{V^2}{2}+\psi$\quad se il flusso è stazionario vale $B=\mathrm{cost}$\\
	\textit{Dim:} la costanza di $B$ si ottiene da (\ref{ns pot}.\sz{ BQ }) introducendo l'ipotesi di stazionarietà.\\
	\textit{Oss:} per la validità del teorema di Bernoulli non è necessario che il fluido sia non viscoso, ma è sufficiente che sia incomprimibile con flusso irrotazionale e stazionario. In altre parole, un fluido incomprimibile con flusso potenziale stazionario è assimilabile ad un fluido non viscoso.
	
	\section{Risoluzione}
	\underline{Def:} sotto le ipotesi di fluido incomprimibile con flusso irrotazionale e stazionario, il sistema di Navier-Stokes si riduce ad un sistema di 3 equazioni scalari indipendenti e disaccoppiate, nelle tre incognite $\phi$(1) da cui si ricava $\vec V$, $p$(1) e $\Theta$(1)\\\mbox{}\\

{\let\clearpage\relax \chapter{Varie forme del bilancio della q.d.m.}
	\begin{tabular}{@{}l|l}
	\quad Equazione&\quad Applicazione\\
	\hline
	$D_t\vec V=\vec f+\frac 1\rho\,\mathrm{div}\,\utilde T$&continuo generico\\
	$D_t(\rho\vec V)+\rho\vec V\mathrm{div}\vec V=\rho\vec f+-\vec\nabla p_\mathrm e+\mathrm{div}\,\utilde\tau$&continuo fluido\\
	$D_t(\rho\vec V)+\rho\vec V\mathrm{div}\vec V=\rho\vec f-\vec\nabla p+\mathrm{div}\,\utilde\tau$&fluido stokesiano\\
	$D_t\vec V=\vec f-\frac{1}{\rho}\vec\nabla p+\nu\nabla^2\vec V$&fluido incomprimibile\\
	$\partial_t\vec V+\vec\omega\times\vec V+\nu\,\mathrm{rot}\,\vec\omega=\vec\nabla(\psi+\frac p\rho+\frac{V^2}{2})$&fluido incomprimibile (vers. con vorticità)\\
	$\partial_t\varphi+\frac{p}{\rho}+\frac{V^2}{2}+\psi=f(t)$&flusso potenziale\\
	$\frac{p}{\rho}+\frac{V^2}{2}+\psi=\mathrm{cost}$&flusso potenziale e stazionario
	\end{tabular}}

%%%%%%%%%%%%%%%%%%%%%%%%%%%%%%%%%%%%%%%%%%%%%%%%%%%%%%%%%%%%%%%%%%%%%%%%%%%%%%%%%%%%%%%%%%%%%%%%%%%%%%%%%%%%%%%%%%%%%%%%%%%%%%%%%%%%%%%%%%%%%%%%%%%%%%%%%%%%%%%%%%%%%%%%%%%%%%%%%%%%%%%%%%%%%%%%%%%%%%%%%%%%%%%%%%%%%%%%%%%%%%%%%%%%%%%%%%%%%%%%%%%%%%%%%%%%%%%%%%%%%%%%%%%%%%%%%%%%%%%%%%%%%%%%%%%%%%%%%%%%%%%%%%%%%%%%%%%%%%%%%%%%%%%%%%%%%%%%%%%%%%%%%%%%%%%%%%%%%%%%%%%%%%%%%%%%%%%%%%%%%%%%%%%%%%%%%%%%%%%%%%%%%%%%%%%%%%%%%%%%%%%%%%%%%%%%%%%%%%%%%%%%%%%%%%%%%%%%%%%%%%	
	
\chapter{Vorticità}
	\section{Problema della condizione di aderenza}	\label{cond aderenza}
	\underline{Def:} in presenza di pareti solide non è posibile avere un flusso completamente irrotazionale. Il problema matematico da risolvere per ottenere il campo di velocità attorno a dei contorni solidi è $\begin{cases}\nabla^2\phi=0\\\vec\nabla\phi|_\mathrm{parete}=0\\\vec\nabla\phi|_\infty=U_\infty\end{cases}$Questo ha soluzione solo se $U_\infty=0$, e tale soluzione è $\phi=\mathrm{cost}\;\Rightarrow\;\vec V\equiv 0$
	
	\section{Sorgente di vorticità}	\label{lastra}
	Considero una lastra piana $L$ di spessore nullo in quiete entro un fluido (viscoso) e tutti i volumi cubici adiacenti alle due pareti della lastra. Immagino che la lastra si metta in moto parallelamente a sé stessa, con velocità $\vec U$, in modo impulsivo.\\Prendo un s.d.r. solidale ad essa (asse $x\parallel -\vec U$), quindi $\vec V(P,t_0)\equiv\vec 0$ e $\vec\omega(P,t_0)\equiv 0$. Al termine del transitorio impulsivo $\vec V(P^\prime,t_0^+)=\vec U\quad\forall P^\prime\in\mathbb{E}^3\!\setminus\!L\qquad\vec V(P^{\prime\prime},t_0^+)\equiv\vec 0\quad\forall P^{\prime\prime}\in L$ per la condizione di aderenza. La circuitazione di velocità sul bordo delle facce ortogonali alla parete (con lato superiore parallelo al flusso) è $\oint_\gamma\vec V(P,t_0^+)\cdot\de \vec s=\mp U\delta x$, il segno tenendo conto delle due pareti. Usando il teorema di Stokes $\mp U\delta x=\iint_\mathcal S \vec\omega(P,t_0^+)\cdot\hat n\,\de S=\bar\omega\delta x\delta z$ dove $\bar\omega$ è il valore medio sul volume cubico. Quindi per soddisfare alla condizione di aderenza, all'inizio del moto si forma una regione di vorticità in prossimità delle pareti (oraria negativa sopra, antioraria positiva sotto), la cui intensità in uno strato di spessore $\delta z$ è $\bar\omega=\mp\frac{1}{\delta z}U$. La quantità di vorticità contenuta entro lo strato superiore/inferiore è $\mp US$ e la quantità globale presente in tutto il campo è nulla. Posso concludere che l'instaurarsi del moto causa l'insorgenza di un sottile strato di vorticità sulle pareti della lastra, e la quantità totale di questa vorticità può essere determinata dalla conoscienza della velocità fuori dallo strato vorticoso.	
	
	\section{Equazione della vorticità}	\label{eq vorticita}
	\underline{Def:} $\frac{D\vec\omega}{Dt}=\vec\omega\cdot\utilde\nabla\vec V+\nu\nabla^2\vec\omega$\\
	\textit{Dim:} applico l'operatore rotore alla (\ref{qdm vort}). I vari termini sono $\mathrm{rot}(\partial_t\vec V)=\partial_t\vec\omega\qquad\mathrm{rot}(\vec\omega\times\vec V)\overset{(1)}{=}\vec\omega\,\mathrm{div}\,\vec V-\vec\omega\cdot\utilde\nabla\vec V-\vec V\,\mathrm{div}\,\vec\omega+\vec V\cdot\vec\nabla\omega=\vec V\cdot\vec\nabla\omega-\vec\omega\cdot\utilde\nabla\vec V\qquad\mathrm{rot}\big(\vec\nabla(\ldots)\big)\overset{(2)}{=}\vec0;\quad\mathrm{rot}(-\nu\,\mathrm{rot}\,\vec\omega)\overset{(3)}{=}\nu\,\mathrm{rot}(-\mathrm{rot}\,\vec\omega)=\nu\,\mathrm{rot}(\nabla^2\vec V)=\nu\nabla^2\vec\omega$. Riconosco il termine $\partial_t\vec\omega+\vec V\cdot\utilde\nabla\vec\omega=D_t\vec\omega$\\
	\textit{Oss:} $(1)$ doppio prodotto vettoriale\quad$(2)$ rotore del gradiente.\quad Il termine $\vec V\cdot\vec\nabla\omega$ non ha un analogo nel bilancio della q.d.m. e, trascurando gli effetti viscosi ($\nu\approx 0$), ha l'effetto di far variare la vorticità. Si vuole ora analizzare questo termine.
	
	\section{Elementi di vorticità}
		\subsection{Linea vorticosa}
		\underline{Def:} linea tangente in ogni suo punto al vettore $\vec\omega$
		\subsection{Tubo vorticoso}
		\underline{Def:} insieme delle linee vorticose racchiuse da una curva chiusa $\gamma$
		\subsection{Intensità di un tubo vorticoso}	\label{intens tubo vort}
		\underline{Def:} $\Gamma=\oint_\gamma\vec V\cdot\de \vec s\quad\gamma$ è il contorno di una sezione
	\section{Teoremi di Helmholtz}
		\subsection{1o teorema}
		\underline{Def:} l'intensità di un tubo di vorticità è indipendente dalla sezione considerata\\
		\textit{Dim:} preso un volume di tubo vorticoso individuato da due sezioni $\Sigma_1$ e $\Sigma_2$, considero il flusso di vorticità attraverso la sua superficie $\Phi_S=\Phi_{\Sigma_1}+\Phi_{\Sigma_2}+\Phi_{\Sigma_\ell}$. Tenendo conto che $\Phi_{\Sigma_\ell}=0$ ottengo $\Phi_S(\vec\omega)=\int_{\Sigma_1}\vec\omega\cdot\hat n_1\,\de S-\int_{\Sigma_2}\vec\omega\cdot\hat n_2\,\de S$. Per il TdD $\Phi_S(\vec\omega)=\int_V\mathrm{div}\,\vec\omega\,\de \upsilon=\int_{}\mathrm{div}(\mathrm{rot}\,\vec V)\,\de \upsilon=0$, quindi $\Phi_{\Sigma_1}=\Phi_{\Sigma_2}$. Se $\gamma_i=\partial\Sigma_i$ allora, tenendo presente (\ref{intens tubo vort}), posso applicare il TdS $\int_{\Sigma_i(\gamma_i)}\vec\omega\cdot\hat n_i\,\de S=\oint_{\gamma_i}\vec V\cdot\de \vec s=\Gamma\quad\forall i$
		\subsection{2o teorema}
		\underline{Def:} le linee di vorticità sono linee materiali (analoghe al VM)
		\subsection{3o teorema}	\label{3 helm}
		\underline{Def:} l'intensità di un tubo di vorticità resta costante durante il suo spostamento sotto l'effetto del flusso\\
		\textit{Dim:} per il 1o teorema l'intensità è costante in un tubo di flusso, quindi basta considerare una sezione qualsiasi. Per il teorema di Stokes $d_t\Gamma=0$.
		
	\section{Vortex stretching e vortex tilting}	\label{vortex}
	Per capire il termine $\vec\omega\cdot\utilde\nabla\vec V$ considero un filamento vorticoso ad asse rettilineo e parallelo a $\vec\omega$, e scompongo la velocità rispetto ad esso: $\vec V=\vec V^\parallel+\vec V^\perp$ con $\vec V^\parallel=\big(\vec V\cdot\mathrm{vers}(\vec\omega)\big)\mathrm{vers}(\vec\omega)$. Considero la componente di $\vec\omega\cdot\utilde\nabla\vec V$ lungo l'asse del filamento, cioè $\big(\vec\omega\cdot\utilde\nabla\vec V\big)\cdot\mathrm{vers}(\vec\omega)=\vec\omega\cdot\vec\nabla[\vec V\cdot\mathrm{vers}(\vec\omega)]=\vec\omega\cdot\vec\nabla V^\parallel$. Questa è diversa da $0$ quando esiste un gradiente della velocità lungo l'asse. A causa di questo gradiente, due sezioni (ortogonali all'asse) traslano lungo l'asse di quantità differenti, in un intervallo $\delta t$. Poiché il fluido è incomprimibile il volume del filamento racchiuso dalle sezioni deve restare costante. In conclusione, se le due sezioni si allontanano allora in filamento si assottoglia, viceversa si ingrossa.\\Inoltre per (\ref{3 helm}) l'intensità del filamento $\Gamma=\omega S$ deve rimanere costante, quindi la vorticità varierà in modo inverso rispetto alla variazione dell'area del filamento, cioè se il filamento si assottiglia (le sezioni si allontanano) allora la vorticità aumenta. La componente di $\vec\omega\cdot\utilde\nabla\vec V$ lungo l'asse del filamento prende il nome di vortex stretching.
	
	Ora considero la componente di $\vec\omega\cdot\utilde\nabla\vec V$ ortogonale all'asse, cioè $\big(\vec\omega\cdot\utilde\nabla\vec V\big)\cdot\mathrm{vers}(\vec\omega)^\perp=\vec\omega\cdot\vec\nabla[\vec V\cdot\mathrm{vers}(\vec\omega)^\perp]=\vec\omega\cdot\vec\nabla V^\perp$. Questa è diversa da 0 quando esiste un gradiente della velocità ortogonalmente all'asse. A causa di questo gradiente, due sezioni (ortogonali all'asse) traslano ortogonalmente all'asse di quantità differenti, in un intervalli $\delta t$. In conclusione, si ha una rotazione dell'asse del filamento e quindi del vettore vorticità. La componente di $\vec\omega\cdot\utilde\nabla\vec V$ ortogonale all'asse del filamento prende il nome di vortex tilting.
	
	In generale, questi due meccanismi agiscono insieme unitamente alla viscosità, e danno un contributo importante alla dinamica della vorticità nei flussi tridimensionali. Tuttavia per flussi bidimensionali $\vec\omega\cdot\utilde\nabla\vec V=\vec 0$, infatti $\vec V=\big(u(x,y),v(x,y),0\big)$ quindi la terza riga e colonna di $[\utilde\nabla\vec V]$ sono nulle. Essendo $\vec\omega$ in direzione $\hat e_3$, il loro prodotto è nullo. L'equazione della vorticità (\ref{eq vorticita}) per un flusso bidimenionale si riduce a $\partial_t\vec\omega+\vec V\cdot\nabla\vec\omega=\nu\nabla^2\vec\omega$ (equazione di diffusione).
	
	\section{Dinamica della vorticità}
	Nel caso della lastra piana (\ref{lastra}), subito dopo la partenza impulsiva di formano due strati di vorticità (di spessore $\delta y$) in corrispondenza delle pareti. Questa vorticità tende a diffondersi in direzione normale alla lastra, poiché il termine convettivo è nullo \big($\vec V(\text{parete})=\vec 0$\big). Una volta diffusasi negli strati più lontani, il termine convettivo si sovrappone a quello diffusivo e la vorticità viene anche trasportata verso valle. Nel frattempo, a causa della condizione di aderenza, continua a formarsi altra vorticità alla parete che avrà lo stesso comportamento appena descritto.\\
	Per una lastra piana, dopo un transitorio in cui avvengono questi fenomeni, si raggiunge una condizione stazionaria in cui la maggior parte della vorticità è concentrata in una zona intorno alla lastra e a valle di essa (a causa della diffusione una quantità di vorticità è penetrata in tutto il campo). Si è interessati a determinare la lunghzza di penetrazione $\delta_p$, intesa come distanza (in direzione normale alla parete) entra cui è contenuta una certa percentuale della vorticità. Questa dipende dal pesi relativi dei termini diffusivo e convettivo. Se $L$ è la lunghezza della lastra e $U$ è la velocità del flusso, come tempo caratteristico prendo $\delta t=\frac LU$ e dalla teoria matematica si sa che $\delta_p\sim\sqrt{\nu\delta t}=\sqrt{\nu\frac LU}$, quindi $\frac{\delta_p}{L}\sim\sqrt{\frac{\nu}{LU}}=\sqrt{\frac{1}{\mathrm{Re}}}$, cioè all'aumentare di $\mathrm{Re}$ diminuisce la lunghezza di penetrazione. Il numero di Reynolds può essere interpretato anche come rapporto tra gli ordini di gradezza dei due termini $\frac{[\vec V\cdot\utilde\nabla\vec\omega]}{[\nu\nabla^2\vec\omega]}=\frac{?}{?}$.\\
	Il campo di moto è quindi formato da uno strato in cui è concentrata la maggior parte della vorticità mentre nel resto del campo il flusso è praticamente irrotazionale. Questi risultati valgono per un corpo aerodinamico. Inoltre, dal punto di vista aeronautico, per il calcolo delle azioni sul corpo si è interessati alla zona di vorticità prossima alla parete.
	
	\section{Fusso ovunque irrotazionale}	\label{flux ov irr}
	Una prima approssimazione per flussi ad alto Re è quella di trascurare completamente lo strato di vorticità e quindi considerare il flusso irrotazionale in tutto il campo di moto, cioè considerare il fluido come non viscoso.\\
	La velocità si determina risolvendo $\nabla^2\varphi=0$ con opportune condizioni iniziali. Infatti avendo trascurato compltamente lo strato di vorticità si sono trascurati i meccanismi responsabili della sua creazione. Tali meccanismi sono conseguenza della condizione di aderenza che perciò perde di significato; una condizione meno forte dell'aderenza è la condizione di non penetrabilità (\ref{cond contorno}) cioè $\vec\nabla\phi\cdot\hat n=\partial_{\hat n}\phi=0$. La condizione all'infinito rimane invariata $\vec\nabla\phi\vert_\infty=\vec U_\infty$. Una volta trovato $\phi$, quindi $\vec V$, la pressione è determinata tramite il teorema di Bernoulli.
		\subsection{Paradosso di d'Alambert}
		Poiché le azioni aerodinamiche si ricavano per integrazione della vorticità presente nel campo di moto, sotto l'ipotesi di flusso potenziale la vorticità è nulla quindi sono nulle anche le forze aerodinamiche. D'altra parte nella pratica, se un corpo è messo in moto rettilineo uniforme entro un fluido potenziale, si sperimentano delle azioni aerodinamiche (i.d. resistenza) su di esso.\\
		Questo paradosso può essere superato ammettendo che le azioni aerodinamiche sono detrminate unicamente dalla vorticità presente nello strato limite e nella scia del corpo. Per corpi aerodinamici in flussi ad alto Re, lo strato limite può considerarsi talmente sottile da non violare l'ipotesi di flusso ovunque irrotazionale.
		Nel caso bidimensionale, il problema (\ref{flux ov irr}) è indeterminato, cioè esistono infinite soluzioni che dipendono dal valore della circuitazione di velocità $\Gamma_0$ lungo una generica curva contenente il corpo. Il problema diventa determinato fornendo tale valore, e inoltre la stima migliore delle azioni aerodinamiche si ha scegliendo un opportuno valore di $\Gamma_0$.
	
	\section{Teorema di Kutta-Joukowski} 	\label{kutta}
	\underline{Def:} $\begin{cases}L=-\rho U\Gamma_0\\D=0\end{cases}$\\
	\textit{Dim:} considero un corpo (aerodinamico) in un flusso potenziale e applico (\ref{bq}) in forma convettiva, tenendo conto dell'ipotesi di flusso potenziale, al dominio $\Upsilon$ (che non è semplicemente connesso) delimitato da una sfera $\Sigma\equiv\Sigma_R$ centrata in $O\in\mathcal C$ e dalla frontiera $\partial\mathcal C$ del continuo: $\int_\Upsilon\partial_t\vec V\,\de \upsilon=\int_{\Sigma\cup\partial\mathcal C}\rho\vec V(\vec V\cdot\hat n)\,\de S=-\int_{\Sigma\cup\partial\mathcal C}p\hat n\,\de S+\int_{\Sigma\cup\partial\mathcal C}\vec\tau_{\hat n}\,\de S$. Utilizzo le ipotesi di flusso potenziale, $\vec f$ trascurabili e tenengo conto della condizione di non penetrazione su $\partial\mathcal C$ ottenendo $\int_\Sigma\rho\vec V\cdot\hat n\,\de S=-	\int_\Sigma p\hat n\,\de S-\int_{\partial\mathcal C}p\hat n\,\de S$. Il secondo termine del RHS rappresenta la risultante delle forze aerodinamiche (dovute al flusso irrotazionale) agenti sul continuo. Poiché il problema è di flusso esterno al corpo, riordinando l'equazione, prendo il limite $\vec F=\lim_{R\to+\infty}[-\int_{\Sigma_R}p\hat n\,\de S-\int_{\Sigma_R}\rho\vec V(\vec V\cdot\hat n)\,\de S]$\\
	\textit{Oss:} nel caso di flussi bidimensionali ad alto Reynolds intorno a corpi aerodinamci, l'ipotesi di flusso potenziale può fornire una stima accettabile (in prima approssimazione) della portanza purché si dia un buon valore di $\Gamma_0$, ma fornisce un valore sempre nullo (quindi una stima errata) della resistenza. Per questo devo tenere conto dello strato di vorticità intorno al corpo e a valle di esso. Sebbene all'interno di tale strato dovrei risolvere le equazioni complete del moto per flussi incomprimibili, si possono ottenere significative semplificazioni grazie alla teoria di Prandtl. \\\mbox{}\\

%%%%%%%%%%%%%%%%%%%%%%%%%%%%%%%%%%%%%%%%%%%%%%%%%%%%%%%%%%%%%%%%%%%%%%%%%%%%%%%%%%%%%%%%%%%%%%%%%%%%%%%%%%%%%%%%%%%%%%%%%%%%%%%%%%%%%%%%%%%%%%%%%%%%%%%%%%%%%%%%%%%%%%%%%%%%%%%%%%%%%%%%%%%%%%%%%%%%%%%%%%%%%%%%%%%%%%%%%%%%%%%%%%%%%%%%%%%%%%%%%%%%%%%%%%%%%%%%%%%%%%%%%%%%%%%%%%%%%%%%%%%%%%%%%%%%%%%%%%%%%%%%%%%%%%%%%%%%%%%%%%%%%%%%%%%%%%%%%%%%%%%%%%%%%%%%%%%%%%%%%%%%%%%%%%%%%%%%%%%%%%%%%%%%%%%%%%%%%%%%%%%%%%%%%%%%%%%%%%%%%%%%%%%%%%%%%%%%%%%%%%%%%%%%%%%%%%%%%%%%%%	

{\let\clearpage\relax \chapter{Teoria dello strato limite di Prandtl}
Nel seguito si supporrà $\vec f(P,t)\equiv\vec 0$
	\section{Strato limite e ipotesi fondamentale} 	\label{def sl}
	\underline{Def:} regione del campo di moto in prossimità di un corpo entro cui non sono trascurabili gli effetti viscosi. La velocità del fluido passa da un valore esterno dovuto al flusso potenziale, al valore 0 dovuto alla condizione di aderenza. Questo strato coincide con lo strato entro cui è contenuta la vorticità.\\ L'ipotesi fondamentale su cui si basa la teoria di Prandtl è $\frac{\delta_p}{\mathsf L}\ll 1$\\
	\textit{Oss:} dire che nello SL gli effetti viscosi non sono trascurabili è equivalente a dire che hanno ordine di grandezza paragonabile a quello degli effetti convettivi, cioè $\mathcal O(\nu\nabla^2\vec V)=\mathcal O(\vec V\cdot\utilde\nabla\vec V)$.
	
	\section{Equazioni di strato limite (o di Prandtl)}	\label{eq prandtl}
	\underline{Def:} $\partial_xu+\partial_yv=0\qquad\partial_tu+u\partial_xu+v\partial_yu=-\frac 1\rho\de _xp+\nu\partial_{yy}u\qquad\partial_yp\approx 0\qquad\omega\simeq\partial_yu\qquad\tau(x,0)=\mu\partial_yu(x,0)$\\
	\textit{Dim:} considero un flusso incomprimibile su una lastra piana, per cui $\vec V=\big(u(x,y),v(x,y),0\big)$ e valuto gli ordini di grandezza dei vari termini assumendo come dimensioni caratteristiche $\delta_p\sim\delta\quad\begin{array}{@{}l}u\sim\mathsf U\\v\sim\mathsf V\end{array}\quad\begin{array}{@{}l}x\sim\mathsf L\\y\sim \delta_p\end{array}\quad\begin{array}{@{}l}t_x\sim\mathsf T_x=\frac{\mathsf L}{\mathsf U}\\t_y\sim\mathsf T_y=\frac{\delta}{\mathsf V}\end{array}$\\
	Per (\ref{bm}) in coord. cartesiane $\partial_xu+\partial_yv=0\;\Rightarrow\;\mathcal O(\partial_xu)=\mathcal O(\partial_yv)\;\Rightarrow\;\frac{\mathsf U}{\mathsf L}=\frac{\mathsf V}\delta\;\Rightarrow\;\mathsf V=\mathsf U\frac{\delta}{\mathsf L}\;\Rightarrow\;\mathsf V\ll\mathsf U$\\
	Per (\ref{bq}) in coord. cartesiane $\begin{pmatrix}\partial_t u\\\partial_tv\end{pmatrix}+\begin{pmatrix}u\\v\end{pmatrix}\cdot\begin{pmatrix}\partial_xu&\partial_xv\\\partial_yu&\partial_yv\end{pmatrix}=-\frac 1\rho\begin{pmatrix}\partial_xp\\\partial_yp\end{pmatrix}+\nu\begin{pmatrix}\partial_{xx}u+\partial_{yy}u\\\partial_{xx}v+\partial_{yy}v\end{pmatrix}$.\\
	Dalla prima comp. $\mathcal O\big(\partial_t u\big)=\frac{\mathsf U^2}{\mathsf L}\quad
\mathcal O\big(u\partial_xu\big)=\frac{\mathsf U^2}{\mathsf L}\quad
\mathcal O\big(v\partial_yu\big)=\frac{\mathsf U\mathsf V}{\delta}=\frac{\mathsf U}{\delta}\mathsf U\frac{\delta}{\mathsf L}=\frac{\mathsf U^2}{\mathsf L}\quad
\mathcal O\big(\nu(\partial_{xx}u+\partial_{yy}u)\big)=\mathcal O\big(\nu\big)\!\cdot\!\big(\frac{\mathsf U}{\mathsf L^2}+\frac{\mathsf U}{\delta^2}\big)\simeq\mathcal O\big(\nu\big)\frac{\mathsf U}{\mathsf L^2}\overset{(\ref{def sl})}{=}\frac{\mathsf U^2}{\mathsf L}$\quad dunque ottengo $\frac{\mathsf U^2}{\mathsf L}=\mathcal O\big(\frac 1\rho\partial_xp\big)+\frac{\mathsf U^2}{\mathsf L}$ quindi deve essere $\mathcal O(\frac 1\rho\partial_xp)=\frac{\mathsf U^2}{\mathsf L}\;\Rightarrow\;\mathcal O(\partial_xp)=\rho\frac{\mathsf U^2}{\mathsf L}$.\\
	Dalla seconda comp. analogamente $\mathcal O\big(\partial_t v\big)=\frac{\mathsf U^2}{\mathsf L}\frac{\delta}{\mathsf L}\quad
\mathcal O\big(u\partial_xv\big)=\frac{\mathsf U^2}{\mathsf L}\frac{\delta}{\mathsf L}\quad
\mathcal O\big(v\partial_yv\big)=\frac{\mathsf U^2}{\mathsf L}\frac{\delta}{\mathsf L}\quad
\mathcal O\big(\nu(\partial_{xx}v+\partial_{yy}v)\big)=\frac{\mathsf U^2}{\mathsf L}\frac{\delta}{\mathsf L}$\quad
dunque ottengo $\frac{\mathsf U^2}{\mathsf L}\frac{\delta}{\mathsf L}=\mathcal O\big(\frac 1\rho\partial_yp\big)+\frac{\mathsf U^2}{\mathsf L}\frac{\delta}{\mathsf L}$ quindi deve essere $\mathcal O(\frac 1\rho\partial_yp)=\frac{\mathsf U^2}{\mathsf L}\frac{\delta}{\mathsf L}\;\Rightarrow\;\mathcal O(\partial_yp)=\mathcal O(\partial_xp)\!\cdot\!\frac{\delta}{\mathsf L}\;\Rightarrow\;\frac{\mathcal O(\partial_yp)}{\mathcal O(\partial_xp)}\ll 1$\\
Da questi risultati posso assumere $\partial_yp\approx 0$ perciò $p\simeq p(x)\;\Rightarrow\;\partial_x p\simeq\de _xp\qquad$\\
	Per la vorticità $\omega=\partial_xv-\partial_yu\;\Rightarrow\;\mathcal O(\omega)=\frac{\mathsf V}{\mathsf L}+\frac{\mathsf U}{\delta}=\frac 1{\mathsf L}\mathsf U\frac{\delta}{\mathsf L}+\frac{\mathsf U}{\delta}\simeq\frac{\mathsf U}{\delta}\;\Rightarrow\;\mathcal O(\omega)\simeq-\partial_yu$\\
	\textit{Oss:} poiché posso trascurare la variazione di pressione in direzione normale alla parete, conoscendo la pressione al bordo dello SL (dal modello di flusso potenziale) posso ricavare la pressione sul corpo proiettando ortogonalmente ad esso la pressione del corrispondente punto sul bordo. Imponendo le condizioni al contorno alla parete $u(x,0)=v(x,0)=0$ e al bordo dello SL $u(x,\delta_p)=U_\mathrm{ext}$, una volta nota la distribuzione della pressione al bordo, posso risolvere le equazioni rispetto a $u$ e $v$, per poi ottenere gli sforzi viscosi $\tau$.
		
	\section{Spessore dello strato limite}
		\subsection{Spessore convenzionale dello strato limite}
		\underline{Def:} distanza $\delta\equiv\delta_p$ dalla parete a cui la velocità tangenziale del fluido raggiunge il 99\% della velocità del flusso potenziale $U_\mathrm{ext}$. Equivalentemente, in cui è contenuto il 99\% della vorticità
		\subsection{Spessore di spostamento}
		\underline{Def:} $\delta^\star=\int_0^\delta\big(1-\frac{u(x,y)}{U_\mathrm{ext}(x)}\big)\de y$\\
		\textit{Dim:} moltiplicando per il flusso di massa $\rho U_\mathrm{ext}$ dovuto al flusso potenziale esterno ottengo $\rho U_\mathrm{ext}\delta^\star=\int_0^\delta\rho(U_\mathrm{ext}-u)\,\de y=\rho U_\mathrm{ext}\delta-\int_0^\delta\rho u\,\de y$ da cui $\rho U_\mathrm{ext}(\delta-\delta^\star)=\int_0^\delta\rho u\,\de y$.\\
		\textit{Oss:} il RHS rappresenta il flusso di massa entro lo strato limite dovuto al flusso effettivo, il LHS rappresenta il flusso di massa entro lo strato limite aumentato[???] di $\delta^\star$ dovuto al flusso potenziale esterno.
		\subsection{Spessore di quantità di moto}
		\underline{Def:} $\theta(x)=\int_0^\delta\frac{u(x,y)}{U_\mathrm{ext}(x)}\big(1-\frac{u(x,y)}{U_\mathrm{ext}(x)}\big)\de y$
		\subsection{Soluzione secondo Blasius}
		\underline{Def:} $u(x,\delta(x))=U_\mathrm{ext}(x)\equiv U_\infty\qquad \delta_{99\%}(x)=\frac{5.2x}{\sqrt{\mathrm{Re}_x}}\quad \delta^\star(x)=\frac{1.72x}{\sqrt{\mathrm{Re}_x}}\quad\theta(x)=\frac{0.664x}{\sqrt{\mathrm{Re}_x}}\qquad c_f(x)=\frac{\tau(x,0)}{\frac 12\rho U_\infty^2}=\frac{0.664}{\sqrt{\mathrm{Re}_x}}$\\
		\underline{Oss:} il punto $X=0$ è una singolarità [DI COSA?] e la teoria di Blasius va in crisi, mentre resta valida $\forall x\in [\varepsilon,L]$ con $\varepsilon\to 0$.
		Considerando che $\mathrm{Re}_x\propto x$ si vede che gli spessori crescono come $\frac{x}{\sqrt x}=\sqrt x$ menre il coeff. come $\frac 1{\sqrt x}$.
			\subsubsection{Autosimilarità}
			Le evidenze sperimentali mostrano che la velocità $U_\mathrm{ext}$ al bordo dello strato limite è pressoché costante in tutti i suoi punti e pari al flusso indisturbato $U_\mathrm{ext}=U_\infty$. Posso quindi pensare che i profili di velocità dei vari punti del corpo sono simili tra loro, cioè che hanno la stessa forma ma sono riscalati in direzione $y$ di un fattore dipendente dalla coordinata $x$. Per fare questo introduco la variabile adimensionale $\eta=\frac{y}{\Delta}$ dove $\Delta(x)=\sqrt{\frac{\nu x}{U_\infty}}$ e cerco soluzioni del tipo $\frac{u(x,y)}{U_\infty}=g(\eta)$. Si può dimostrare che tale problema si riconduce ad una eq. diff. ord. la cui soluzione (numerica) fornisce il profilo teorico di velocità.
		\subsection{Azioni aerodinamiche sulla lastra piana}
		\underline{Def:} $L=0\qquad D=2\rho U^2\Theta(L)$\\
		\textit{Dim:} per definizione la portanza è $L=-\iint_\mathtt L -p\hat n\cdot\hat \jmath\,\de S+\iint_\mathtt L\vec\tau\cdot\hat\jmath\,\de S$ e i termini sono nulli perché $-p\hat n\cdot\hat\jmath$ è una funzione simmetrica mentre $\vec\tau\perp\hat\jmath$. Per la resistenza $D=\iint_\mathtt L -p\hat n\cdot\hat\imath\,\de S+\iint_L \vec\tau\cdot\hat\imath\,\de S$ e il primo termine è nullo perchè $\hat n\perp\hat\imath$ mentre per il secondo $\iint_\mathtt L\vec\tau\cdot\hat\imath\,\de S=2\int_0^L\tau(x)\,\de x=2\int_0^L\frac 12\rho U^2c_f(x)\,\de x$ e definendo un coeff. di attrito medio $C_f=\frac 1L\int_0^Lc_f(x)\,\de x=\frac{1.328}{\sqrt{\mathrm{Re}}}$ ottengo $D=\frac{1.328\rho U^2 L}{\sqrt{\mathrm{Re}}}=2\rho U^2\Theta(L)$
	
	\section{Strato limite su pareti curve}
	Le equazioni di Prandtl ricavate per la lastra piana in moto parallelamente a sé stessa continuano a valere (in un s.d.r. locale) se la lastra ha un raggio di curvatura $\rho$ non troppo piccolo. Infatti nella componente di \sz{(BQ)} in direzione normale compare un termine di accelerazione centripeta con ordine di grandezza $\frac{\mathsf U^2}{\rho}$ e affiché sia comparabile con gli altri termini deve essere $\rho\sim\frac{\mathsf L}{\delta}$
	Lontano dal corpo le linee di corrente sono orizzontali e possono essere pensate come delle pareti in quanto esse non possono essere attraversate dal fluido. Si crea una zona (1) a monte del punto di massimo spessore in cui $\de _xU_\mathrm{ext}(x)>0$ (quindi $\de _xp_\mathrm{ext}(x)<0$) e una zona (2) a valle in cui $\de _xU_\mathrm{ext}(x)<0$ (quindi $\de _xp_\mathrm{ext}(x)>0$).\\
Per il th. di Bernoulli \big($\partial_x\{p_\mathrm{ext}+\frac{\rho U_\mathrm{ext}^2}{2}\}=0$\big) si può affermare $\partial_xp_\mathrm{ext}>0\;\Leftrightarrow\;U_\mathrm{ext}\!\cdot\!\partial_xU_\mathrm{ext}<0$, e dato che $U_\mathrm{ext}>0$, questo è equivalente a $\partial_xp_\mathrm{ext}>0\;\Leftrightarrow\;\partial_xU_\mathrm{ext}<0$. Poiché è plausibile assumere $\mathrm{sign}(\partial_xU_\mathrm{ext})=\mathrm{sign}(\partial_xu)$ si può riscrivere tale relazione tenendo conto di (\ref{eq prandtl}.\sz{(BM)}), dopo averla integrata in $y$, ottendo $\partial_xp_\mathrm{ext}<0\;\Leftrightarrow\;v(\delta)<0$.\\
Per queste relazioni, nelle zone con gradiente favorevole lo SL cresce meno velocemente rispetto alla lastra piana, mentre nelle zone con gradiente avverso lo SL cresce più velocemente.	
		\subsection{Forma del profilo di velocità}
		\underline{Def:} $\partial_{yy}u(x,0)=\frac 1\mu\de _xp_\mathrm{ext}(x)$\\
		\textit{Dim:} la (\ref{eq prandtl}.\sz{(BQ)}) valutata in $y=0$ diventa $0=-\frac 1\rho\de _xp_\mathrm{ext}(x)+\nu\partial_{yy}u(x,0)$ cioè $\partial_{yy}u(x,0)=\frac 1\mu\de _xp_\mathrm{ext}(x)$\\
		\textit{Oss:} alla generica coordinata $x$, la curvatura del profilo di velocità alla parete dipende dal gradiente di pressione in quel punto.
%\begin{center}\includegraphics[scale=.4]{prof-vel-1.jpg}\qquad\includegraphics[scale=.4]{prof-vel-2.jpg}\qquad\includegraphics[scale=.4]{prof-vel-3.jpg}\end{center}
		\subsection{Forma del profilo di vorticità}
		\underline{Def:} $\partial_y\omega(x,0)=-\frac 1\mu\partial_{yy}u(x,0)$\\
		\textit{Dim:} valutando l'espressione della vorticità (\ref{eq prandtl}) in $y=0$ ottengo $\partial_y\omega(x,0)\simeq-\partial_{yy}u(x,0)$ per cui $\partial_y\omega(x,0)=-\frac 1\mu\partial_{yy}u(x,0)$\\
		\textit{Oss:} la tangente al profilo di vorticità, alla generica coordinata $x$, è proporzionale all'opposto del gradiente di pressione in quel punto. È opportuno notare che nel caso $\de _xp>0$ la tangente ha inclinazione negativa e di conseguenza, dovendo essere $\omega(x,\delta)=0$, il profilo di vorticità ha punto di flesso a cui corrisponde il massimo valore di vorticità in quel punto.

	\section{Separazione dello strato limite}
	In presenza di un gradiente di pressione avverso $\de _xp_\mathrm{ext}(x)>0$ si verifica un progressivo rallentamnto delle particelle al bordo dello SL che si trasmette anche alle particelle interne perché $\partial_xp=\de _xp_\mathrm{ext}$. Se il gradiente avverso è sufficientemente potente (intenso oppure applicato per un tratto lungo), le particelle fluide adiacenti alla parete (che risentino maggiormente del rallentamento in quanto hanno velocità minore delle altre) potrebbero arrestarsi ed invertire il moto (separazione naturale). In questa situazione lo strato limite non è più aderente al corpo e si individua il punto in cui avviene il distacco (p. di separazione). In 2D, il punto di separazione è quello in cui il profilo di velocità parte con tangente verticale alla parete $\mathrm \partial_yu(x,0)=0$; se avviene la separazione, questa può avvenire solo nella zona (2) in cui c'è un gradiente avverso. A valle della separazione tende a formarsi una scia vorticosa non stazionaria ed il modello di strato limite + flusso potenziale non è più valido.\\
	La presenza di uno spigolo provoca sempre l'immediata separazone dello SL perché lo spigolo è equivalente a un gradiente di pressione tendente all'infinito (separazione forzata) in quanto il flusso dovrebbe cambiare direzione istantaneamente; anche per corpi con elevata curvatura si ha un gradiente di pressione molto elevato.
		\subsection{Corpi aerodinamici}
		Sulla base della separazione dello SL divido i corpi in due cateorie:\\
		corpi aerodinamici: lo SL è completamente attaccato su tutta la superficie. Lasciano una scia sottile formata dall'unione dei due SL. Hanno una forma allungata in direzione del flusso e non devono presentare spigoli nella parte anteriore, mentre la parte posteriore deve essere aguzza\\
		corpi tozzi: lo SL si separa lungo la superficie. Lasciano una scia vorticosa spessa e non stazionaria. La dimensione trasversale al flusso è rilevante e possono avere spigoli nella parte anteriore. 
		
	\section{Risltati della teoria dello strato limite}
	\begin{enumerate}
		\item{la pressione agente sul corpo è la proiezione in direzione normale di quella corrispondente sul bordo dello SL}
		\item{la pressione al bordo la calcolo usando il modello di flusso potenziale}
		\item{devo conoscere il profilo di velocità per calcolare le azioni tangenziali}
		\item{è necessario conoscere la posizione del bordo dello strato limite}
	\end{enumerate}
	
	\section{Interpretazione energetica della resistenza}
	Mi metto in un s.d.r. solidale con il fluido e considero un profilo alare che si muove verso sinistra con velocità $U$. In corrispondenza dell'ascissa $x$ l'energia cinetica (specifica all'ascissa) del fluido è $\kappa(x,t)=\frac 12\rho\int_{-\infty}^{+\infty}\big(u(x,y,t)^2+v(x,y,t)^2\big)\de y$. La presenza del corpo causa la perturbazione della velocità locale del fluido quindi la variazione dell'energia cinetica in direzione $x$. In particolare, a monte del corpo essa tende a $0$ mentre a valle tende a un valore $K_0$ a causa della presenza della scia. In un flusso stazionario, un osservatore solidale al corpo vede la stessa cosa quindi posso pensare che il corpo si porti dietro l'andamento dell'energia cinetica $K(x)$. Dopo un intervallo $\Delta t$ il corpo si è spostato di $U\Delta t$ e posso applicare il teorema dell'energia meccanica per il fluido: $K(t)=\int_{-\infty}^{+\infty}\kappa(x,t)\,\de x\quad K(t+\Delta t)=\int_{-\infty}^{+\infty}\kappa(x,t+\Delta t)\,\de x$. L'unica forza che compie lavoro è la resistenza del corpo sul fluido $-D\hat\imath$, e compie il lavoro $\mathscr L_D =-D(-U\Delta t)=DU\Delta t$. La differenza tra le energia cinetiche è $\mathcal E(t+\Delta t)-\mathcal E(t)=E_0U\Delta t$, cioè pari all'energia contenuta nella scia che si è aggiunta.}\\\mbox{}\\

{\let\clearpage\relax \chapter{Stima iterativa delle azioni aerodinamiche}
Quanto esposto è valido sono per corpi aerodinamici cioè con SL aderente.
	\subsubsection{Passo 0}
	Si risolve il problema di flusso ovunque irrotazionale (\ref{flux ov irr}) $\begin{cases}\nabla^2\phi=0\\\partial_n\phi\vert_{\text{corpo}}=0\\\Gamma=\Gamma_0\end{cases}$ ricavando $V^{(0)}(x,0)$ e $p^{(0)}(x,0)$
	\subsubsection{Passo 1}
	Si risolvono le equazioni di Prandtl (\ref{eq prandtl}) $\begin{cases}\partial_xu+\partial_yv=0\\\partial_tu+u\partial_xu+v\partial_yu=-\frac 1\rho\de _xp^{(0)}(x,0)+\nu\partial_{yy}u\\u(x,0)=v(x,0)=0\\u(x,\delta)=V^{(0)}(x,0)\end{cases}$ assumendo come condizioni al contorno i risultati ottenuti al passo 0, per ricavare il profilo di velocità nello strato limite $u^{(1)}(x,y)$ e $v^{(1)}(x,y)$ con cui si calcola lo spessore di spostamento ${\delta^\star}^{(1)}$.
	\subsubsection{Passo 2}
	Si risolve nuovamente il problema di flusso potenziale (\ref{flux ov irr}) considerando lo spessore di spostamento calcolato al passo 1, per ricavare $V^{(2)}(x,0)$ e $p^{(2)}(x,0)$.s
	\subsubsection{Passo 3}
	Si risolvono nuovamente le equazioni di Prandtl (\ref{eq prandtl}) assumendo come condizioni al contorno i risultati ottenuti al passo 2, per ricavare il profilo di velocità nello strato limite $u^{(2)}(x,y)$ e $v^{(3)}(x,y)$ con cui si calcola lo spessore di spostamento ${\delta^\star}^{(3)}$.
	\subsubsection{Passi successivi}
	Si reiterano i passi precedenti fino al raggiungimento della condizione di convergenza cioè, fissata a priori una tolleranza $\epsilon$, finché le quantità coinvolte subiscono variazioni (tra un'iterazione e la seguente) inferiori a $\epsilon$. Quando l'algoritmo arriva a convergenza di ottiene come risultato (a meno della tolleranza fissata) la posizione del bordo dello SL, la distribuzione di pressione sul corpo e le azioni tangenziali agenti su di esso. Di conseguenza si ottengono (a meno della tolleranza) le stime precise di portanza e resistenza sul corpo.
}
	
%%%%%%%%%%%%%%%%%%%%%%%%%%%%%%%%%%%%%%%%%%%%%%%%%%%%%%%%%%%%%%%%%%%%%%%%%%%%%%%%%%%%%%%%%%%%%%%%%%%%%%%%%%%%%%%%%%%%%%%%%%%%%%%%%%%%%%%%%%%%%%%%%%%%%%%%%%%%%%%%%%%%%%%%%%%%%%%%%%%%%%%%%%%%%%%%%%%%%%%%%%%%%%%%%%%%%%%%%%%%%%%%%%%%%%%%%%%%%%%%%%%%%%%%%%%%%%%%%%%%%%%%%%%%%%%%%%%%%%%%%%%%%%%%%%%%%%%%%%%%%%%%%%%%%%%%%%%%%%%%%%%%%%%%%%%%%%%%%%%%%%%%%%%%%%%%%%%%%%%%%%%%%%%%%%%%%%%%%%%%%%%%%%%%%%%%%%%%%%%%%%%%%%%%%%%%%%%%%%%%%%%%%%%%%%%%%%%%%%%%%%%%%%%%%%%%%%%%%%%%%%	

\chapter{Profili alari}
Un profilo alare sostanzialmente funziona grazie alla conservazione della vorticità in tutto il campo di moto.
%\begin{center}\includegraphics[scale=0.8]{profilo-alare.png}\end{center}
	\section{Elementi di un profilo}	\label{elem prof}	
			\subsubsection{Bordo d'attacco (leading edge)}
			\subsubsection{Bordo d'uscita (trailing edge)}
			\subsubsection{Corda (chord)}
			\underline{Def:} segmento congiungente il bordo d'attacco con il bordo d'uscita
			\subsubsection{Spessore (thickness)}
			\underline{Def:} massima distanza $t$ (perpendicolare alla corda) fra dorso e ventre\\
			\textit{Oss:} rispetto alla lunghezza $c$ della corda, i profili sottili hanno $t\leq 6\%$, i profili semi-spessi hanno $6\%<t<12\%$, i profili spessi hanno $t\geq 12\%$
			\subsubsection{Incidenza geometrica (angle of attack)}
			\underline{Def:} angolo $\alpha$ tra direzione del flusso asintotico e direzione della corda
			\subsubsection{Linea media (camber line)}
			\underline{Def:} luogo dei punti equidistanti dal dorso e dal ventre
			\subsubsection{Freccia (camber)}
			\underline{Def:} massima distanza $f$ (perpendicolare alla corda) fra linea media e corda\\
			\textit{Oss:} se $f=0$ il profilo è simmetrico, altrimenti è curvo.
			\subsubsection{Incidenza di portanza nulla (zero-lift angle)}
			\underline{Def:} incidenza geometrica $\alpha_0$ a cui il profilo non sviluppa portanza\\
			\textit{Oss:} l'angolo $\alpha_0$ individua l'asse di portanza nulla. In altre parole $alpha_0$ è l'angolo tra corda e asse di portanza nulla.
			\subsubsection{Incidenza aerodinamica}
			\underline{Def:} angolo $(\alpha-\alpha_0)$ tra direzione del flusso asintotico e asse di portanza nulla
			\subsubsection{Centro aerodinamico}
			\underline{Def:} punto situato a distanza $\frac c4$ dal BA in cui si pensano applicate le forze aerodinamiche, cioè ripetto cui si annulla $\vec M$.
		\subsubsection{Denominazione NACA}
		\underline{Def:} NACAxyzz\quad x-freccia (in centesimi di corda)\quad y-distanza del punto di freccia dal BA\quad zz-spessore (in \% di corda)\\
		\textit{Oss:} i profili della serie NACA00zz sono simmetrici.
		\section{Risultante delle azioni}
		\underline{Def:} la forza risultante (per unità di lunghezza) delle azioni agenti su un profilo è equivalente al sistema $\{(C,\vec F)+\vec M^\mathrm{(C)}\}$
		\subsection{Coefficieni aerodinamici}	\label{coeff aero}
		\underline{Def:}\begin{tabular}{llll}
			\textbf{c. di portanza}&\textbf{c. di resistenza}&\textbf{c. di momento}&\textbf{c .di pressione}\\
			$C_\mathrm L=\frac{L}{\frac 12\rho U_\infty^2c}$&$C_\de =\frac{D}{\frac 12\rho U_\infty^2c}$&$C_\mathrm M=\frac{M}{\frac 12\rho U_\infty^2c^2)}$&$C_\mathrm p=\frac{p-p_\infty}{\frac 12\rho U_\infty^2}=1-\frac{V^2}{U_\infty^2}$
		\end{tabular}\\
		\textit{Oss:} per il coeff. di pressione, il suo massimo valore è $C_P=1$ assunto se $V=0$, cioè nei punti di ristagno. È utile introdurre i coefficienti aerodinamici perché così si supera il problema della conversione tra unità di misura, e in secondo luogo è possibile applicare la similitudine meccanica.
			
    \section{Distribuzione di velocità e pressione su un cilindro circolare}
    Un cilindro circolare sia immerso trasversalmente in un flusso potenziale. Le linee del flusso asintotico sono parallele fra loro a distanza sufficiente dal corpo. Una di esse impatta il cilindro nel suo punto più avanzato rispetto alla direzione del flusso terminando su di esso (p. di ristagno anteriore). La particella fluida che segue tale linea decelera dalla velocità asintotica $\vec U_\infty$ alla velocità nulla nel PR anteriore $\vec V_\mathrm o^{(\mathrm{ant})}=\vec 0$. Allontanandosi dal PR anteriore, la particella accelera seguendo il profilo del cilindro fino a una velocità massima (locale?) $\vec V_\mathrm{picco}$ che viene assunta in un punto corrispondente allo spessore massimo (picco di aspirazione). Continuando a seguire il profilo del corpo, la particella decelera fino alla velocità nulla assunta nel punto di ristagno posteriore $\vec V_\mathrm o^{(\mathrm{post})}=\vec 0$, dopodiché si stacca dal corpo riallineandosi alla corente asintotica.\\
    Per il th. di Bernoulli la pressione ha un andamento opposto a quello della velocità, cioè aumenta dal valore asintotico $p_\infty$ fino al PR anteriore in cui si ha $p_\mathrm o>p_\infty$. Continuando a seguire il profilo la pressione diminuisce fino al picco di aspirazione in cui si ha un minimo (locale?) per poi aumentare fino al PR posteriore in cui si ha ancora $p_\mathrm o>p_\infty$, dopodiché la particella si stacca per riallinearsi alla corrente asintotica.
        \subsection{Distribuzione su un profilo alare}
        Si può estendere questa analisi a un profilo alare considerando che il raggio di curvatura non è costante e nemmeno uniforme tra dorso e ventre. È proprio questo squilibrio geometrico che produce una asimmetria della distribuzione di velocità, e quindi di pressione, tra dorso e ventre causando una differenza di pressione. Per stimare la distribuzione di pressione agente su un profilo alare si usa il $C_p$ nella sua definizione in termini di velocità (\ref{}) valutata su un certo numero di punti di controllo situati lungo la superficie. La distribuzione qualitativa tipica è riportata in figura.

\begin{center}
\begin{tabular}{cc}
%	\includegraphics[scale=.5]{distrib-cp.png}&
%	\includegraphics[scale=.5]{plot-cp.png}\\
Distribuzione qualitativa $C_P$ &
Grafico $C_P$-$\frac xc$
\end{tabular}
\end{center}

    \section{Genesi della portanza}
	Considero un profilo simmetrico ad incidenza nulla investito da un flusso potenziale. Le azioni tangenziali sono di entità ridotta, sono rilevanti soprattutto le azioni di pressione. Finché $\alpha=0$ le azioni sul dorso e sul ventre sono equilibrate quindi non si sviluppa nessuna portanza. Inclindando il profilo con un angolo $\alpha>0$ si verifica la presenza di una forza di portanza.
	
	Ora il profilo è messo in moto (impulsivamente) con incidenza $\alpha\neq 0$ (questo equivale a considerare qualsiasi corpo \sz{(bidimensionale)} aerodinamico); si sperimenta una forza di portanza che vale $L\overset{(\ref{kutta})}{=}-\rho U\Gamma_0$, e si ha $L>0$ quando $\Gamma_0<0$. Al tempo iniziale si ha $\vec\omega(0^+)=0$ quindi $\Gamma=\oint\vec\omega\cdot\hat n\de S=0$ e valgono le equazioni di flusso potenziale (\ref{}). Simultaneamente per la condizione di aderenza si generano gli strati di vorticità. A causa dell'elevato gradiente avverso localizzato nello spigolo al BU, lo SL inferiore (contenente vorticità antioraria posiva) non riesce a girare attorno ad esso senza separarsi. Dopo un transitorio, la vorticità contenuta dello SL di ventre si separa dal profilo e viene immessa sul flusso sotto forma di un vortice (v. di partenza), che viene trasportato dal flusso (irrotazionale) a valle del profilo. Questo vortice antiorario positivo determina una circuitazione di velocità positiva $+\Gamma$ per qualche curva chiusa esterna al vortice. Per la conservazione della vorticità totale (rif. ???) una circuitazione negativa $-\Gamma$ deve essere presente nello spazio ove il flusso è non-irrotazionale, cioè nello SL e nella scia.
	
	Si può dimostrare che la vorticità contenuta della scia è nulla. Per le (\ref{eq prandtl}) la vorticità nella scia (unione degli SL) è $\omega_\mathrm{scia}=\int_{\delta^-}^{\delta^+}-\partial_yu\,\de y=u(\delta^-)-u(\delta^+)=U_\mathrm{ext}^--U_\mathrm{ext}^+$. Per vedere che $\omega_\mathrm{scia}=0$ applico il th. di Bernoulli a un circuito costituito da 2 linee di corrente sopra e sotto il profilo ($1^+\!\to\!2^+$ e $1^-\!\to \!2^-$), 1 segmento a monte del profilo che le unisce ($1^+\!\to\!1^-$), 1 segmento a valle del profilo che attraversa la scia ($2^+\!\to\!2^-$). Dunque valgono $B(1^+)=B(2^+)\quad B(1^-)=B(2^-)$ perché sono linee di corrente e $B(1^+)=B(1^-)$ perché è immerso nel flusso potenziale. Di conseguenza vale $B(2^+)=B(2^-)$ cioè $p(2^+)+\frac 12\rho U^2(2^+)=p(2^-)+\frac 12\rho U^2(2^-)$, per ogni sezione trasversale della scia. Ricordando le (\ref{eq prandtl}) vale $p(\delta^+)=p(\delta^-)$ quindi $p(2^+)=p(2^-)$ e di conseguenza $U(2^+)=U(2^-)$ cioè $U_\mathrm{ext}^+=U_\mathrm{ext}^-$. Quindi la vorticità nella scia è $\omega_\mathrm{scia}=0$ dunque la circuitazione $-\Gamma$ deve essere contenuta negli SL.
	
	Nello SL di ventre si ha vorticità positiva mentre nello SL di dorso si ha vorticità negativa. D'altra parte globalmente si deve avere una circolazione negativa, quindi si deve per forza avere un surplus di vorticità negativa nello SL di dorso. Questo sistema di vorticità è equivalente ad un vortice di intensità $\iint\omega\,\de S<0$ applicato nel baricentro del sistema.

        Posso pensare la soluzione del flusso come somma del contributo irrotazionale e del contributo indotto dal vortice $\vec V=\vec V_\mathrm{irr}+\vec V_\mathrm{ind}$. Considerando che $\vec V_\mathrm{ind}(P)\perp\overrightarrow{OP}$, l'effetto della velocità indotta, cioè del vortice e quindi della vorticità negli SL, è quello di accelerare le particelle che transitano sopra il bordo dello SL di dorso e decelerare le particelle che transitano sotto il bordo dello SL di ventre. Perciò globalmente si ha $U_\mathrm{dorso}>U_\mathrm{ventre}$ e per il th. di Bernoulli $p_\mathrm{dorso}<p_\mathrm{ventre}$. La differenza di pressione fra ventre e dorso è causa di una forza (f. di portanza) di verso concorde.
	
        L'intensità del vortice di partenza, quindi della circolazione che si crea intorno al profilo, è una funzione crescente della velocità asintotica e dell'angolo di incidenza. Se uno di questi parametri viene variato durante il moto, si ha la formazione (al BU) di un altro vortice concentrato (di signo opportuno)  con conseguente variazione di circolazione e quindi di portanza. Se il profilo venisse arrestato impulsivamente, si formerebbe (al BU) un vortice concentrato uguale ed opposto al vortice di partenza.
        
	%[Nota simpatica sull'applicazione del th di stokes al dominio s. connesso]\\
	
	\section{Strato limite e condizione di Kutta}
	Se in un profilo alare immerso in un flusso si pensa di schiacciare a zero lo spessore dello strato limite (ciò equivale a trascurare del tutto gli effetti della viscosità) si ottiene la situazione limite in cui il profilo è immerso in un flusso potenziale, di cui le superfici di dorso e ventre rappresentano due linee di flusso. Inoltre, la scia si riduce a una linea di corrente che si stacca dal BU, quindi il flusso risultante è caratterizzato dal fatto che nessuna linea di corrente gira intorno al BU, cioè il PR posteriore coincide con il BU. Questa condizione viene chiamata condizione di Kutta.
	
	Si può dimostrare che, nel flusso reale, la circolazioene di velocità $+\Gamma$ intorno al profilo (localizzata nello SL di dorso per quanto trovato al n. precedente) è determinata imponendo la condizione di Kutta. Infatti il problema (rif. flusso irr con cond scorrimento) ammette infinite soluzioni dipendenti dal valori della circolazione di velocità lungo una generica curva che contiene il profilo, e la condizione di Kutta individua quella per cui non si hanno linee di corrente che girano intorno al BU. Di conseguenza il teorema di Kutta-Joukowski fornisce la miglior stima della portanza 
	
	\section{Teoria dei profili sottili}
	Per capire meglio come la portanza dipende quantitativamente dall'incidenza si può far ricorso ad una procedura di calcolo semplificata.
		\subsection{Ipotesi}	\label{prof sott}
		\begin{enumerate}[nosep]
			\item{flusso irrotazionale con condizione di Kutta (trascuro SL)}
			\item{profilo sottile}
			\item{curvatura piccola}
			\item{incidenza piccola}
		\end{enumerate}
		\subsection{Teorema di Glauert}
		\underline{Def:} la portanza è data dalla somma di 3 contributi $L=L^{(1)}+L^{(2)}+L^{(3)}$ dovuti alla decomposizione del problema del flusso attorno a un profilo in tre problemi semplici.\\
		\textit{Dim:} vedi numeri successivi\\
		\textit{Oss:} $L^{(1)}$ è dovuta al flusso attorno a un profilo simmetrico dello stesso spessore. $L^{(2)}$ è dovuta al flusso intorno alla linea media a incidenza nulla e non dipende dall'incidenza. $L^{(3)}$ è dovuta al flusso intorno alla corda con stessa incidenza ed è direttamente proporzionale all'incidenza.
		\subsection{Velocità e potenziale di perturbazione}
		\underline{Def:} $\vec V(x,y)=\vec U_\infty+\vec V(x,y)_\mathrm p\qquad\vec\nabla\varphi_\mathrm p=\vec V_\mathrm p$\\
		\textit{Dim:} posso pensare di decomporre il campo di velocità $\vec V(x,y)=\vec U_\infty+\vec V_\mathrm p(x,y)$ introducendo una velocità di perturbazione dovuta alla presenza del corpo. Per l'ipotesi di flusso irrotazionale il potenziale di velocità soddisferà $\vec\nabla\varphi=\vec U+\vec V_\mathrm p$ quindi posso definire il potenziale di perturbazione oltre quello di flusso indisturbato.
		\subsection{Equazioni del problema}
		\underline{Def:} $\begin{cases}\nabla^2\varphi_\mathrm p=0\\\vec\nabla\varphi_\mathrm p|_\infty=\vec 0\\(\vec U_\infty+\vec\nabla\varphi_\mathrm p)\cdot\hat n=0\end{cases}$\\
		\textit{Dim:} $\nabla^2\varphi=\nabla^2\varphi_\infty+\nabla^2\varphi_\mathrm p=\nabla^2\varphi_\mathrm p$ perché $\varphi_\infty$ è costante nello spazio. All'infinito deve valere $\vec\nabla\varphi|_\infty=\vec\nabla\varphi_\infty|_\infty+\vec\nabla\varphi_\mathrm p|_\infty\overset{!}{=}\vec U$ quindi $\vec\nabla\varphi_\mathrm p|_\infty=0$. Per la condizione di aderenza, note le curve $\gamma_d=\big(x,y^+(x)\big)$ e $\gamma_v=\big(x,y^-(x)\big)$ che parametrizzano il dorso e ventre rispettivamente, deve valere $\vec V(\gamma_i)\cdot\hat n(\gamma_i)=0$\\
		\textit{Oss:} la condizione al contorno sul corpo è non lineare.
		\subsection{Linearizzazione della condizione al contorno}
		La condizione posso scriverla $\big(U\cos\alpha+u_\mathrm p(\gamma_i)\big)n_x(\gamma_i)+\big(U\sin\alpha+v_\mathrm p(\gamma_i)\big)n_y(\gamma_i)=0$ da cui $v_\mathrm p(\gamma_i)=-(U\cos\alpha+u_\mathrm p)\frac{n_x(\gamma_i)}{n_y(\gamma_i)}-U\sin\alpha$. Per il rapporto $\frac{n_x}{n_y}$, ricordando che $(\de _sx,\de _sy )$ è il vettore tangente a una curva, vale $\frac{n_x(\gamma_i)}{n_y(\gamma_i)}=-\frac{\de y^\pm}{\de x}$. Quindi ho ottenuto $v_\mathrm p(\gamma_i)=\frac{\de y^\pm}{\de x}\big(U\cos\alpha+u_p(\gamma_i)\big)-U\sin\alpha$. Per le definizioni (\ref{elem prof}) posso scrivere $y^\pm(x)=y_{LM}(x)\pm y_s(x)$ e ottenere così una decomposizione in più sottoproblemi, applicando in seguito le ipotesi (\ref{prof sott}).
			\subsubsection{Flusso intorno al profilo simmetrico}
			$v_\mathrm p^{(1)}\equiv v_\mathrm p\big(x,\pm y_s(x))=\pm\frac{\de y_s(x)}{\de x}\big(U\cos\alpha+u_\mathrm p(x,\pm y_s(x))\big)\simeq\pm\frac{\de y_s(x)}{\de x}\big(U+u_\mathrm p(x,\pm y_s(x))\big)\simeq\pm\frac{\de y_s(x)}{\de x}U$
			\subsubsection{Flusso intorno alla linea media}
			$v_\mathrm p^{(2)}\equiv v_\mathrm p(x,y_{LM})=\frac{\de y_{LM}}{\de x}\big(U\cos\alpha+u_\mathrm p(x,y_{LM}(x))\big)\simeq\frac{\de y_{LM}}{\de x}\big(U+u_\mathrm p(x,y_{LM}(x))\big)\simeq\frac{\de y_{LM}}{\de x}U$
			\subsubsection{Flusso intorno alla corda}
			$v_\mathrm p^{(3)}\equiv v_\mathrm p(x,0^\pm)=-U\sin\alpha\simeq -U\alpha$
			\subsubsection{Flusso totale}
			$v_\mathrm p^{(1)}+v_\mathrm p^{(2)}+v_\mathrm p^{(3)}$
			
		\subsection{Risultati fondamentali}	\label{risult prof sott}
		\begin{enumerate}
			\item{La portanza è lineare nell'incidenza\qquad$L=L^{(2)}+L^{(3)}=L_0+k\alpha\quad k$ dipende dalla curvatura}
			\item{La portanza è applicata nel centro aerodinamico e il momento rispetto ad esso non dipende da $\alpha$}
			\item{Il coeff. di portanza dipende dall'incidenza aerodinamica\qquad$C_L=C_{L,\alpha}(\alpha-\alpha_0)\quad C_{L,\alpha}\!=\!\frac{\de C_L}{\de \alpha}\!=\!\begin{cases}2\pi&\alpha\text{ in radianti}\\0.11&\alpha\text{ in gradi}\end{cases}$}
			\item{$\alpha_0$ dipende dalla curvatura\qquad$\alpha_0=\frac 1\pi\int_0^\pi\frac{\de y_{LM}}{\de x}(1-\cos\theta)\,\de \theta$\quad con il cambiamento $x=\frac c2(1-\cos\theta)$}
	\end{enumerate}
	\textit{Oss:} per profili simmetrici ovviamente $\alpha_0=0$ mentre per profili con curvatura in alto (basso) vale $\alpha_0>0\;\;(<0)$\\A parità di curvatura, si sperimenta un aumento di $|\alpha_0|$ se il punto di massima curvatura è vicino al BU \big(vd. (\ref{flap})\big)\\
		\subsection{Conclusioni sull'effetto dello spessore}
		Nonstante si siano trascurati gli effetti dello SL e dello spessore, i risultati ottenuti sono in eccellente accordi con i risultati sperimentali, purché l'incidenza non assuma valori troppo elevati. In realtà lo spessore e la presenza dello SL hanno due effetti opposti che tendono a compensarsi. Infatti all'aumentare dello spessore la portanza tende ad aumentare e questo causa l'aumento dei picchi di aspirazione sul dorso \sz{(rispetto a un profilo sottile)} e quindi del gradiente avverso. Di conseguenza lo SL sul dorso è più spesso \sz{(rispetto a un profilo sottile)} e questo riduce la curvatura e l'incidenza effettiva perciò diminuisci anche la portanza.

	\section{Curva $C_L$-$\alpha$}
%\begin{center}\includegraphics[scale=0.75]{curva-cl-a.png}\end{center}
	Il risultato n. 3 di (\ref{risult teor prof sott})  permette di graficare l'andamento del $C_L$ con l'incidenza ottenendo un segmento retta con pendenza 0.11. Questo risultato è valido per incidenze non troppo elevate.
Per profili simmetrici il segmento passa per l'origine essendo $\alpha_0=0$, mentre per profili curvi viene traslato verso l'alto.\\
L'effetto dello spessore è pressoché trascurabile perché 

	\section{Stallo}
%\begin{wrapfigure}{r}{0\textwidth} \includegraphics[scale=0.5]{stallo.jpg}\end{wrapfigure}
	All'aumentare dell'incidenza aumenta anche il picco di aspirazione sul dorso del profilo, e di conseguenza il gradiente avverso di pressione presente dal picco al BU. Continuando ad aumentare l'incidenza, si raggiunge una condizione in cui lo sl sul dorso comincia a separarsi. Dopo la separazione, la portanza cresce meno velocemente (non si ha più adamento lineare) e la resistenza comincia ad aumentare. Per un certo angolo di incidenza (a. di stallo $\alpha_{st}$) la portanza raggiunge il suo massimo valore $\big(C_L^{max}=C_L(\alpha_{st})\big)$, dopo del quale comincia a diminuire a causa della massiccia separazione dello SL sul dorso. In questa situazione il profilo si dice stallato.\\
	La dinamica dello stallo e le caratteristiche di un profilo stallato sono proprietà legate alla separazione dello SL (quindi al gradiente di pressione) nonché alla resistenza alla separazione (quindi al regime nello SL).
		\subsection{Effetto del Re}
		Nella zona lineare si è trascurato lo SL quindi l'effetto del Re non si considera. Alle alte incidenze, all'aumentanre del Re il p.to di transizione si sposta verso il BU e quindi lo stallo avviene ad incidenze maggiori permettendo di raggiungere $C_L^{max}$ più elevati.
		\subsection{Effetto della curvatura}
		A parità di $C_L$, i profili curvi \sz{(verso il basso)} hanno una distribuzione di pressione tale che il gradiente avverso sul dorso è meno intenso rispetto ai profili simmetrici e più spostato verso il BU. Per questo un profilo curvo raggiunge un certo valore di $C_L$ in corrispondenza di un' incidenza minore di quella necessaria per un profilo simmetrico.
%\begin{wrapfigure}{r}{0\textwidth} \includegraphics[scale=0.6]{stallo-spessori.jpg}\end{wrapfigure}
		\subsection{Effetto dello spessore}	
		Per profili della stessa famiglia, al diminuire dello spessore diminuisce il raggio di curvatura al BA. Quindi il picco di aspirazione si sposta verso il BA aumentando il gradiente avverso a valle. Di conseguenza la separazione è anticipata ad incidenze minori limitando il $C_L^{max}$ raggiungibile.
			\subsubsection{Profili sottili}
			Hanno un elevato gradiente avverso in prossimità del BA che causa la separazione dello SL ancora in regime laminare. Dopo la separazione si ha la transizione in turbolento e, se il gradiente avverso non è troppo elevato, può succedere che lo SL si riattacca verso valle formando una bolla di separazione sul dorso. All'aumentare dell'incidenza la bolla di separazione cresce gradualmente fino ad occupare tutto il dorso, quindi lo stallo è graduale. Sono caratterizzati da $\alpha_{st}$ e $C_L^{max}$ più bassi rispetto ai profili spessi.
			\subsubsection{Profili semispessi}
			Anche questi sono caratterizzati dalla formazione della bolla di separazione in prossimità del BA. Tuttavia, all'aumentare dell'incidenza le dimensioni della bolla non variano apprezabilmente fino ad una incidenza critica, alla quale la bolla scoppia e il flusso separato occupa tutto il dorso provocando uno stallo improvviso del profilo.
			\subsubsection{Profili spessi}
			La separazione inizia vicino al BU, in quanto si ha un gradiente avverso più moderato. All'aumentare dell'incidenza il punto di separazione si sposta verso il BA, finché il flusso separato, che è turbolento e non può riattaccarsi, occupa tutto il dorso provocando uno stallo graduale.
		
	\section{Resistenza}
	Poiché non è possibile usare la teoria dei profili sottili o altre teorie semplificate, cioè non è possibile trascurare lo SL, bisogna ricorrere a dati sperimentali. Le condizioni di resistenza minima, di solito, sono vicine a quelle di incidenza ideale, cioè l'incidenza alla quale il PR anteriore coincide con il BA.
		\subsection{Curva $C_L$-$C_D$ (o polare)}
%\begin{center}\includegraphics[scale=.75]{curva-polare.jpg}\end{center}
		\underline{Def:} $C_D=C_{D0}+kC_L^2-C_{LM}^2\qquad C_{D0}=C_D(\alpha\!=\!0)$\\
		\textit{Oss:} $C_{D0}, k$ e $C_{LM}$ sono parametri assegnati per ogni profilo. $C_{LM}$ è il valore di $C_L$ per cui si ha la minima resistenza.
		\subsection{Contributi alla resistenza globale}
		\underline{Def:} $D=D_\mathrm{att}+D_p=\int\vec\tau\cdot\mathrm{vers}(\vec U_\infty)\,\de S+\int -p\hat n\cdot\mathrm{vers}(\vec U_\infty)\,\de S$\\
		\textit{Oss:} alle basse incidenze $\hat n\cdot\mathrm{vers}(\vec U_\infty)\approx 0$ quindi $D_\mathrm{att}\gg D_p$
			\subsubsection{Resistenza di attrito}
			Per un corpo in movimento immerso in un fluido viscoso, le particelle a contatto con esso sono in moto con il corpo ed esercitano un'azione accelerante sulle particelle più prossime ad esso. Per il principio di azione e reazione, queste particelle esercitano un'azione frenante sul corpo, a causa delle azioni tangenziali che nascono in seguito al gradiente di velocità in direzione normale al corpo. Tale forza dipende dalla superficie del corpo bagnata fluido, dalla sua velocità e dalla viscosità, ma non dalla forma del corpo.
			\subsubsection{Resistenza di forma (o di pressione)}
			Se il fluido non fosse viscoso le linee di corrente sarebbero simmetriche rispetto al corpo, generando campi di velocità e pressione identici tra monte e valle. A causa della viscosità il fluido perde energia \big(vd. (\ref{be})\big) aggirando il corpo e ciò causa la separazione delle linee di flusso con conseguente formazione di una zona di ricircolazione nella regione posteriore del corpo. A causa di questo la sezione di passaggio del fluido si restringe, la velocità deve perciò aumentare mentre la pressione diminuisce. Essendo la zona di ricircolazione in equilibrio con il flusso a valle, la pressione del fluido a valle è minore di quella a monte, e questa differenza genera la forza della resistenza di forma. Essa è fortemente dipendente dalla forma del corpo perché le linee di flusso seguono il suo contorno.
		\subsection{Effetto dell'incidenza}
		Alle basse incidenze, domina la resistenza di attrito perché il contributo di forma è piccolo finché lo SL è sottile e attaccato. Inoltre, essa non varia significativamente con l'incidenza, così come la resistenza totale.\\		
		Alle alte incidenze, il contributo della resistenza di forma cresce in concomitanza al progredire del distaccamento dello SL, fino a diventare dominante. La resistenza totale aumenta significativamente con l'incidenza.
		\subsection{Effetto della turbolenza}
		Poiché uno SL turbolento sviluppa una resistenza d'attrito maggiore di uno laminare, alle basse incidenze questo ha un effetto negativo sulla resistenza totale. Alle alte incidenze invece la turbolenza ha un effetto positivo perché sposta il punto di separazione dello SL verso il BU, riducendo la resistenza di pressione e quindi la resistenza totale.
		\subsection{Effetto della curvatura}
		La curvatura verso il basso ha l'effetto di spostare il valore $C_D^{min}$ in corrispondenza di $C_L$ più alti, quindi un effetto fortemente benefico.
		\subsection{Effetto dello spessore}
		Alle basse incidenze,con l'aumento dello spessore si ha l'aumento della resistenza a causa dello spessore della scia.\\
		Alle alte incidenze, la resistenza cresce più velocemente per profili sottili, perché per essi il distacco dello SL avviene a $C_L$ inferiori.

	\section{Efficienza aerodinamica}
	\underline{Def:} $E=\frac LD=\frac{C_L}{C_D}=\frac{C_L}{C_{D0}+kC_L-C_{LM}^2}$\qquad sulla polare $E=\tan\beta$\\
	\textit{Oss:} annullando $\frac{\de E}{\de C_L}=0$ ottengo che $\max E$ viene assunto per $C_L=\sqrt{\frac{C_{D0}}{k}+C_{LM}^2}$

	\section{Momento aerodinamico e punti notevoli di un profilo}
	La forza aerodinamica risultante che si sviluppa su un profilo genera un momento aerodinamico rispetto al BA. Si può dimostrare che può essere approssimato con il momento della componente normale alla corda, che è distante $\frac c4$ dal BA.\\
	Alle basse incidenze, il momento aerodinamico rispetto a tale punto è pressoché costante con l'incidenza (0 per profili simmetrici, negativo per profili curvi verso il basso).\\
	Alle alte incidenze, all'inizio del distacco dello SL non è più costante e tende ad essere negativo. Ad ogni modo, il comportamento alle alte incidenze è legato alle modalità di stallo (stallo brusco e variazione brusca, ecc).

%%%%%%%%%%%%%%%%%%%%%%%%%%%%%%%%%%%%%%%%%%%%%%%%%%%%%%%%%%%%%%%%%%%%%%%%%%%%%%%%%%%%%%%%%%%%%%%%%%%%%%%%%%%%%%%%%%%%%%%%%%%%%%%%%%%%%%%%%%%%%%%%%%%%%%%%%%%%%%%%%%%%%%%%%%%%%%%%%%%%%%%%%%%%%%%%%%%%%%%%%%%%%%%%%%%%%%%%%%%%%%%%%%%%%%%%%%%%%%%%%%%%%%%%%%%%%%%%%%%%%%%%%%%%%%%%%%%%%%%%%%%%%%%%%%%%%%%%%%%%%%%%%%%%%%%%%%%%%%%%%%%%%%%%%%%%%%%%%%%%%%%%%%%%%%%%%%%%%%%%%%%%%%%%%%%%%%%%%%%%%%%%%%%%%%%%%%%%%%%%%%%%%%%%%%%%%%%%%%%%%%%%%%%%%%%%%%%%%%%%%%%%%%%%%%%%%%%%%%%%%%	


\chapter{Ipersostentazione}
	\section{Motivazioni}
	Nel regime di volo livellato in condizioni di crociera, l'equilibrio verticale del velivolo restituisce $W=L_\mathrm{cr}$ in cui la portanza è calcolata nelle condizione di crociera $L_\mathrm{cr}=\frac 12\rho_\mathrm{cr}U_\mathrm{cr}^2SC_L$. Da qui posso determinare (a meno di $C_L$) la superficie alare necessaria per sviluppare tale portanza $(SC_L)_\mathrm{cr}=\frac{W}{\frac 12\rho_\mathrm{cr}U_\mathrm{cr}^2}$.\\
	In regime di decollo, l'equilibrio verticale resituisce $W+Ma_z=L_\mathrm{to}$ in cui la portanza è calcolata nelle condizioni di decollo $L_\mathrm{tol}=\frac 12\rho_\mathrm{tol}U_\mathrm{tol}^2SC_L$ e l'inirzia verticale è stimata (in prima approssimazione) a circa 20\% del peso. Da qui posso determinare (a meno di $C_L$) la superficie alare necessaria per sviluppare tale portanza $(SC_L)_\mathrm{tol}=\frac{W+Ma_z}{\frac 12\rho_\mathrm{tol}U_\mathrm{tol}^2}=\frac{1.2W}{\frac 12\rho_\mathrm{tol}U_\mathrm{tol}^2}$.\\
	La necessita dell'ipersostentazione nasce dal fatto che $U_\mathrm{tol}\ll U_\mathrm{cr}$ perciò $(SC_L)_\mathrm{tol}\gg(SC_L)_\mathrm{cr}$. Poiché è impraticabile fissare un valore di $C_L$ e determinare la superficie alare di conseguenza (ala sovradimensionata per la maggior parte del volo e sviluppo eccessivo di resistenza d'attrito) l'unica alternativa è dimensionare la superficie in base alle condizioni di crociera ed intervenire sul profilo in fase di decollo/atterraggio per ottenere il $C_L$ necessario alla sostentazione del velivolo in aria.
	
	\section{Dispositivi di ipersostentazione}
	L'ipersostentazione si applica per aumentare $C_L^max$ nelle fasi di decollo e atterraggio, fino ad arrivare a valori $3.5\div 4$. Tutti i dispositivi di ipersostentazione aumentano la resistenza rispetto al profilo pulito.
		\subsection{Dispositivi integrati nell'ala}
		Non modificano la geometria del profilo e hanno lo scopo di ritardare (ad incidenze più alte) lo stallo del profilo agendo sulle zone più a rischio di separazione. Vengono praticati dei canali in cui sono alloggiati dei dispositivi che permettono di energizzare le particelle dello SL. Tuttavia tali dispositivi aumentano il peso e rendono difficile la manutenzione, per cui in genere non vengono utilizzati
			\subsubsection{Soffiaggio}
			Si impiegano delle pompe che iniettano soffi d'aria ad alta velocità in direzione parallela alla parete, aumentando la velocità delle particelle e quindi ritardando la separazione dello SL. La curva $C_L$-$\alpha$ viene spostata verso $\alpha_\mathrm{st}$ e $C_L^max$ maggiori.
			\subsubsection{Aspirazione}
			Si impiegano dispositivi che che aspirano gli strati di fluido vicini alla parete (quindi più lenti) in modo da spostare quelli più esterni (quindi più veloci) verso la parete ritardando la separazione. La curva $C_L$-$\alpha$ viene spostata verso $\alpha_\mathrm{st}$ e $C_L^max$ maggiori.
		\subsection{Dispositivi modificatori di geometria}
		 Inoltre, i flaps incrementano il valore del momento aerodinamico. Questo è un effetto negativo perché richiede di equilibrare il velivolo creando deportanza nella coda. Di conseguenza si ha una piccola riduzione dell'aumento di portanza in ala ottenibile con l'ipersostentazione, oltre all'aumento della resistenza.
			\subsubsection{Flaps}	\label{flap}
%\begin{wrapfigure}{r}{0\textwidth} \includegraphics[scale=.8]{flap.jpg}\end{wrapfigure}
			Modificano la geometria al BU. Sono superfici estraibili che permettono la rotazione della parte posteriore del profilo aumentandone la curvatura al BU. Per quanto detto al (\ref{risult prof sott}) questo implica l'aumento del $C_L^{max}$ ma la riduzione di $\alpha_\mathrm{st}$, per questo non si riesce a raggiungere un elevato valore di $C_L^{max}$. Inoltre non è possibile aumentare troppo la curvatura perché altrimenti la curva $C_L$-$\alpha$ si sposta troppo verso sinistra diminuendo $\alpha_\mathrm{st}$.	Per ovviare a questo in genere si realizza un gap tra i flaps estratti e il profilo, in modo che la differenza di pressione tra dorso e ventre ha un effetto di soffiaggio locale (si ha una separazione locale sul flap in cui la velocità aumenta e la pressione diminuisce). Inoltre, spesso vengono utilizzati in combinazione con modificatori di BA.
			\subsubsection{Slats}
			Modificano la geometria al BA. Sono superfici estraibili che creano un allungamento (estrazione e rotazione) del profilo alare addolcendo il picco di aspirazione vicino al naso del profilo (utile per profili che stallano di naso). Contemporaneamente, si crea una fessura che permette l'aspirazione da ventre a dorso, con effetto di soffiaggio, nonché un aumento della curvatura al BA (con effetti trascurabili). Aumenta l'incidenza di stallo e il $C_L^{max}$. [METTERE GRAFICI]
			\subsubsection{Slots}
			Modificano la geometria al BA. Sono composti da meccanismi che permettono di spostare il naso del profilo creando una fessura che permette l'aspirazione da dorso a ventre aumentando $\alpha_\mathrm{st}$ quindi il $C_L^{max}$ raggiungibile. La velocità del soffiaggio (e quindi l'efficacia) dipende dalla differenza di pressione tra dorso e ventre, e il massimo si ottiene in corrispondenza del picco di aspirazione. In genere vengono utilizzati in concomitanza ai flaps. La curva $C_L$-$\alpha$ viene allungata [METTERE DISEGNO].
%\begin{center}\begin{tabular}{cc}\includegraphics[scale=.7]{slat.jpg}&\includegraphics[scale=.5]{slot.jpg}\\Slat&Slot\end{tabular}\end{center}
			
%%%%%%%%%%%%%%%%%%%%%%%%%%%%%%%%%%%%%%%%%%%%%%%%%%%%%%%%%%%%%%%%%%%%%%%%%%%%%%%%%%%%%%%%%%%%%%%%%%%%%%%%%%%%%%%%%%%%%%%%%%%%%%%%%%%%%%%%%%%%%%%%%%%%%%%%%%%%%%%%%%%%%%%%%%%%%%%%%%%%%%%%%%%%%%%%%%%%%%%%%%%%%%%%%%%%%%%%%%%%%%%%%%%%%%%%%%%%%%%%%%%%%%%%%%%%%%%%%%%%%%%%%%%%%%%%%%%%%%%%%%%%%%%%%%%%%%%%%%%%%%%%%%%%%%%%%%%%%%%%%%%%%%%%%%%%%%%%%%%%%%%%%%%%%%%%%%%%%%%%%%%%%%%%%%%%%%%%%%%%%%%%%%%%%%%%%%%%%%%%%%%%%%%%%%%%%%%%%%%%%%%%%%%%%%%%%%%%%%%%%%%%%%%%%%%%%%%%%%%%%%	

\chapter{Ali ad apertura finita}
Il s.d.r. viene preso con origine nella mezzeria dell'apertura alare, con asse $x$ in direzione e verso della velocità asintotica, asse $y$ lungo l'apertura alare, asse $z$ in direzione verticale verso l'alto. La distanza fra le estremità alari , cioè l'apertura alare, è $b$.\\
%\begin{wrapfigure}{r}{0\textwidth}\includegraphics[scale=0.7]{ala-finita.jpg}\end{wrapfigure}
	\section{Definizioni geometriche}
			\subsubsection{Forma in pianta}
			\underline{Def:} forma dell'ala vista dall'alto
			\subsubsection{Superficie alare}
			\underline{Def:} area $S=\int_{-c/2}^{c/2}c(y)\,\de y$ della proiezione della forma in pianta sul piano $xy$\\
			\textit{Oss:} la superficie totale (dorso e ventre) è notevolemnte più grande della superficie alare
			\subsubsection{Corda geometrica}
			\underline{Def:} media $\bar c=\frac bS$
			\subsubsection{Allungamento alare}
			\underline{Def:} $A\!R=\frac{b^2}{S}$\\
			\textit{Oss:} è una misura di quanto l'ala è allungata (in relazione alla corda). Per ali di elevato A\!R, il meccanismo con cui si genera la circuitazione e la portanza è qualitativamente simile al caso dei profili.
			\subsubsection{Incidenza aerodinamica}
			\underline{Def:} angolo $\alpha_W$ tra $\vec U_\infty$ e asse di portanza nulla
			\subsubsection{Rastremazione}
			\underline{Def:} $\lambda=\frac{c(\pm\frac b2)}{c(0)}$
			\subsubsection{Svergolamento}
			\underline{Def:} gli assi di portanza nulla dei profili non sono tutti paralleli\\
			\textit{Oss:} se l'ala ha svegolamento nullo e profili uguali allora $\alpha_w(y)=\alpha_w=\text{cost}$.
			
	\section{Flusso trasversale, vorticità assiale e velocità indotta}
	Se l'ala è portante, la differenza di pressione fra dorso e ventre ha l'effetto di creare un flusso di particelle attraverso le estremità alari nel verso coerente. Questo flusso trasversale è sovrapposto al flusso longitudinale per cui le linee di corrente sull'ala sono deviate rispetto alla direzione asintotica \sz{(sul ventre verso le estremità, sul dorso verso la mezzeria)}.\\
	A causa della condizione di aderenza, nel flusso trasversale si crea una vorticità assiale $\omega_x$ (i cui tubi vorticosi sono paralleli all'asse di mezzeria) che si sovrappone alla vorticità trasversale $\omega_y$ \sz{(che genera portanza)}. Si crea uno SL trasversale \sz{(in realtà una componente trasversale di velocità nello SL)}. La nuova vorticità assiale $\omega_x$, a differenza di quella trasversale, ha lo stesso segno negli SL di dorso e ventre (e quindi anche nella scia). Schiacciando a zero lo spessore dello SL posso trascurare l'effetto della vorticità trasversale ma non di quella assiale, infatti presa una sezione alare $A$ i contributi assiali di dorso e ventre sono concordi e si sommano in un vortice concentrato  di intensità $\Gamma_x^A$ che ha il suo opposto nella sezione $B$ coniugata \sz{(cioè simmetrica rispetto alla mezzeria)} e di intensità $\Gamma_x^B=-\Gamma_x^A$. Poiché questi vortici si trovano ad una distanza finita non possono annullarsi e inducono una velocità non trascurabile in tutti i punti del profilo e della scia, che ha principalmente componente verso il basso (downwash). Questa velocità indotta agisce sulla scia dell'ala deformandola verso il basso e cambiando la distribuzione di vorticità in essa. Inoltre la velocità indotta causa la separazione dello SL trasversale in corrispondenza delle estremità alari. La combinazione di questi due effetti ha la conseguenza di concentrare la vorticità della scia in due nuclei posti alle estremità alari (vortici di estremità). Tuttavia per ali ad elevato A\! R questi vortici possono essere considerati come disturbi localizzati.
%\begin{center}\includegraphics[scale=0.75]{sl-trasv.jpg}\includegraphics[scale=0.8]{vort-assi.jpg}\end{center}
	
	\section{Teoria della linea portante}
		\subsection{Ipotesi}	\label{hp linea}
		\begin{enumerate*}[label=$(\roman*)$]
			\item{ala dritta con $\mathrm{A\!R}\geq 7$}
			\item{ogni sezione alare è in  flusso irrotazionale}
			\item{spessore nullo degli strati limite}
			\item{variazione trascurabile della componente trasversale di velocità indotta}
		\end{enumerate*}
		\subsection{Linea portante}
		Poiché lo spessore degli SL è schiacciato a zero, i vortici assiali \sz{(dovuti alla velocità indotta)} sono ridotti a dei filamenti vorticosi semi-infiniti, che si staccano dal BU dell'ala e sono tutti paralleli alla velocità asintotica. La vorticità trasversale (a cui è dovuta la portanza) viene rappresentata prolungando i filamenti vorticosi coniugati per formare un insieme di vortici \sz{(ciascuno di intensità costante)} a ferro di cavallo lungo l'apertura alare. Questi vortici vengono poi condensati in un segmento di linea vorticosa di intensità variabile $\Gamma(y)$, chiamata linea portante. Essa rappresenta la circuitazione presente intorno alla sezione di apertura $y$.
%\begin{center}\includegraphics{ferri-vort.jpg}\includegraphics[scale=1.4]{linea-portante.jpg}\end{center}
%\includegraphics[scale=0.6]{vel-indotta.jpg}\includegraphics[scale=.8]{forza-ala.jpg}
		\subsection{Velocità indotta e incidenza indotta}	\label{vel ind}
		\underline{Def:} $w_\mathrm{ind}(y)=\frac{1}{4\pi}\int_{-b/2}^{b/2}\frac{\gamma(y)}{\eta-y}\,\de \eta\qquad\tan\alpha_\mathrm{int}\simeq\alpha_\mathrm{ind}=\frac{w_\mathrm{ind}(y)}{U_\infty}\qquad w_\mathrm{ind}(y)\simeq\alpha_\mathrm{ind}(y)U_\infty\quad U_\mathrm{eff}\simeq U_\infty$\\
		Fissata una sezione alare di apertura $y$, la velocità indotta dalla linea portante sulla sezione (che per ipotesi ha solo componente verticale) è data dalla somma di tutti i contributi indotti dai filamenti vorticosi sulla sezione considerata.  Ognuno di questi contributi è pari alla metà di quello che darebbe un filamento infinito, che è dato dalla formula di Biot-Savart $w_\mathrm{ind}(y)=\frac{1}{4\pi}\int_{-b/2}^{b/2}\frac{\gamma(y)}{\eta-y}\,\de \eta$.\\
		La velocità effettiva $\vec U_\mathrm{eff}(y)$ vista dal profilo è la risultante della velocità asintotica e della velocità indotta sulla sezione $\vec U_\mathrm{eff}(y)=\vec U_\infty-w_\mathrm{ind}(y)\,\hat k$, perciò il profilo lavora ad un'incidenza effettiva $\alpha_\mathrm{eff}$ diversa da quella dovuta al solo flusso asintotico. Si può definire l'incidenza indotta come $\tan\big(\alpha_\mathrm{ind}(y)\big)=\frac{w_\mathrm{ind}(y)}{U_\infty}$ per cui l'incidenza effettiva è $\alpha_\mathrm{eff}(y)=\alpha(y)-\alpha_\mathrm{ind}(y)$. Tuttavia l'angolo di incidenza indotta è piuttosto piccolo e si può assumere $\tan\alpha_\mathrm{int}\simeq\alpha_\mathrm{ind}$ da cui le importanti relazioni $w_\mathrm{ind}(y)\simeq\alpha_\mathrm{ind}(y)U_\infty$ e $U_\mathrm{eff}\simeq U_\infty$.
		\subsection{Equazione della linea portante}
		\underline{Def:} $\Gamma(y)=\frac 12C_{L\alpha}(y)c(y)\big[U_\infty\alpha(y)-\frac{1}{4\pi}\int_{-b/2}^{b/2}\frac{\de \Gamma}{\de y}(y)\frac{\de \eta}{y-\eta}\big]$\\
		\textit{Dim:} nell'espressione della velocità indotta (\ref{vel ind}), l'intensità $\gamma(y)$ dei vari filamenti vorticosi può essere legata alla circuitazione $\Gamma(y)$ della linea portante. Prese due sezioni $A$ e $B$ relative alle coordinate $y_1$ e $y_2$, considero due curve chiuse $\mathcal C_1$ e $\mathcal C_2$ attorno a tali sezioni e la superficie $\mathcal S$ compresa fra di esse. Applico il TdS introducendo un taglio (a contributo nullo) su $\mathcal S$ per cui $\int_\mathcal S\vec\omega\cdot\hat n\,\de S=\oint_{\partial\mathcal S}\vec V\cdot\de \vec s\;\Rightarrow\;\int_{y_1}^{y_2}\gamma(y)\,\de y=\Gamma(\mathcal C_1)-\Gamma(\mathcal C_2)=\Gamma_1-\Gamma_2$. Nel limite in cui $B\to A$, cioè $y_2=y_1+\de y$, vale $\Gamma_2=\Gamma_1+\frac{\de \Gamma}{\de y}\,\de y$ per cui ottengo $\gamma(y)=-\frac{\de \Gamma}{\de y}(y)$. Sostituisco questa relazione nell'espressione  della vel. indotta ottenendo $w_\mathrm{ind}(y)=\frac{1}{4\pi}\int_{-b/2}^{b/2}\frac{1}{y-\eta}\frac{\de \Gamma}{\de y}(y)\,\de \eta$, in cui $y$ è la sezione alare in cui sto valutando la \sz{(componente verticale della)} velocità indotta, $\eta$ è la variabile di integrazione che individua il filamento vorticoso, $\Gamma(y)$ è la vorticità trasversale (che genera portanza) presente intorno alla sezione alare considerata.\\
Voglio esprimere $w_\mathrm{ind}(y)$ in funzione di $\Gamma(y)$ per avere quest'ultima come unica incognita. Per fare ciò considero la portanza sviluppata da una sezione alare $\delta L(y)\overset{(\ref{coeff aero})}{=}\frac 12\rho U_\mathrm{eff}^2(y)c(y)C_{L\alpha}(y)\alpha_\mathrm{eff}(y)\overset{(\ref{kutta})}{=}\rho U_\mathrm{eff}(y)\Gamma(y)$ da cui ottengo $\Gamma(y)=\frac 12C_{L\alpha}(y)c(y)U_\mathrm{eff}(y)\big(\alpha(y)-\alpha_\mathrm{ind}(y)\big)$. Considerando le (\ref{vel ind}) ottengo $\Gamma(y)=\frac 12C_{L\alpha}(y)c(y)\big[U_\infty\alpha(y)-\frac{1}{4\pi}\int_{-b/2}^{b/2}\frac{\de \Gamma}{\de y}(y)\frac{\de \eta}{y-\eta}\big]$, che è un'equazione integro-differenziale nell'incognita $\Gamma(y)$. Questa può essere risolta facilmente per ottenere la distribuzione della circuitazione sull'ala [IN CHE MODO?].
		\subsection{Forze aerodinamiche}
		\underline{Def:} $\delta L(y)\simeq\rho U_\infty\Gamma(y)\quad\delta D(y)\simeq\rho U_\infty\Gamma(y)\alpha_\mathrm{ind}\qquad L=\rho U_\infty\int_{-b/2}^{b/2}\Gamma(y)\,\de y\quad D_\mathrm{ind}=\rho U_\infty\int_{-b/2}^{b/2}\alpha_\mathrm{ind}(y)\Gamma(y)\,\de y$\\
		\textit{Dim:} una volta nota la distribuzione della vorticità assiale lungo la linea portante $\Gamma(y)$, per il th. di Kutta-Joukowski la forza $\delta\vec F(y)$ agente sulla generica sezione ha intensità $\delta F(y)=\rho U_\infty\Gamma(y)$ e direzione $\delta\vec F\perp\vec U_\mathrm{eff}$ perché ho trascurato lo SL.\\
		Dunque posso decomporre questa forza rispetto a $\vec U_\infty$ ottenendo una componente di resistenza indotta dalla vorticità assiale dovuta al flusso trasversale $\delta\vec F=\delta\vec L+\delta\vec D_\mathrm{ind}$ in cui, tenendo conto della (\ref{vel ind}), $\delta L(y)=\delta F(y)\cos\alpha_\mathrm{ind}\simeq\delta F(y)\quad\delta D_\mathrm{ind}(y)=\delta F(y)\sin\alpha_\mathrm{ind}\simeq\delta F(y)\alpha_\mathrm{ind}$.\\
		Le forze risultanti agenti sull'ala sono $L=\int\delta L(y)\,\de y=\rho U_\infty\int_{-b/2}^{b/2}\Gamma(y)\,\de y$ e $D_\mathrm{ind}=\int\delta D_\mathrm{ind}(y)\,\de y=\rho U_\infty\int_{-b/2}^{b/2}\alpha_\mathrm{ind}(y)\Gamma(y)\,\de y=\rho\int_{-b/2}^{b/2}w_\mathrm{ind}(y)\Gamma(y)\,\de y$\\
		\textit{Oss:} la resistenza indotta è un contributo caratteristico delle ali ad apertura finita che va ad aggiungersi alla resistenza di profilo $D_w=D_p+D_\mathrm{ind}=D_\mathrm{att}+D_p+D_\mathrm{ind}$. Inoltre, può anche essere spiegata da considerazioni energetiche infatti, schiacciando a zero gli SL, non si può comunque trascurare la vorticità assiale di scia poichè questa contiene una certa quantità di energia cinetica che fa variare l'energia totale del fluido mentre l'ala si sposta; deve quindi esistere una forza che compie un lavoro a cui imputare tale variazione.
		
	\section{Distribuzione ellittica della circuitazione}	\label{istr ell}
		\underline{Def:} $\Gamma(y)=\Gamma_0\sqrt{1-(\frac{2y}{b})^2}=\Gamma_0\sin\theta$\quad minimizza $D_\mathrm{ind}\qquad\begin{array}{l}w_\mathrm{ind}(y)=\frac{\Gamma_0}{2b}\\\alpha_\mathrm{ind}(y)=\frac{\Gamma_0}{2U_\infty b}\end{array}\quad\begin{array}{l}C_L=\frac{\pi b\Gamma_0}{2US}\\C_{D_\mathrm{ind}}=\frac{\pi\Gamma_0^2}{4U_\infty^2S}\end{array}\quad C_{D_\mathrm{ind}}=\frac{C_{D_\mathrm{ind}}}{\pi\text{A\!R}}$\\
	\textit{Dim:}  [DIMOSTRA CHE ALPHA IND E W IND SONO COSTANTI]\\
	$L=\rho U_\infty\int_{-b/2}^{b/2}\Gamma(y)\,\de y=\frac{\rho U_\infty\pi b\Gamma_0}{4}\;\Rightarrow\;C_L=\frac{\pi b\Gamma_0}{2US}\qquad D_\mathrm{ind}=\alpha_\mathrm{ind}L=\frac{\Gamma_0}{2U_\infty b}\frac{\rho U_\infty\pi b\Gamma_0}{4}=\frac{\pi\rho\Gamma_0^2}8\;\Rightarrow\;C_{D_\mathrm{ind}}=\frac{\pi\Gamma_0^2}{4U_\infty^2S}$\\$C_{D_\mathrm{ind}}=kC_L^2\;\Rightarrow\; k=\frac{\pi\Gamma_0^2}{4U_\infty^2S}\frac{4U_\infty^2S^2}{\pi^2b^2\Gamma_0^2}=\frac{S}{\pi b^2}=\frac 1{\pi\text{A\!R}}$
		\subsection{Distribuzioni non ellittiche}
		\underline{Def:} $C_{D_\mathrm{ind}}=\frac{C_L^2}{\pi A\!R}(1+\delta)=\frac{C_L^2}{e\pi A\!R}\qquad\delta=0\:(e=1)$ per distribuzione ellittica\\
		\textit{Oss:} si può dimostrare che per $\lambda\in[0.2,0.4]$ si ha $\delta\approx 0$
		
	\section{Curva $C_L$-$\alpha_w$ per distribuzione ellittica}
	\underline{Def:} per ali a svergolamento nullo e profili uguali $C_{L\alpha}^w=\frac{\pi b}{2U_\infty S}\frac{\de C_L}{\de\alpha_w}\quad C_{L\alpha}^w=C_{L\alpha}^o\,\frac{\mathrm{A\!R}}{\mathrm{A\!R}+2}\qquad C_L^w=C_{L\alpha}^w\alpha_w$\\
	\textit{Dim:} per (\ref{distr ell}) vale $C_{L\alpha}^w=\frac{\mathrm dC_L}{\de\alpha_w}=\frac{\pi}2\frac{b}{U_\infty S}\frac{\de\Gamma_0}{\de\alpha_w}$. La portanza sviluppata dall'ala è $L=\frac 12\rho U_\infty^2\int_{-b/2}^{b/2}c(y)C_{L\alpha}^o(\alpha-\alpha_\mathrm{ind})\,\de y=\frac 12\rho U_\infty^2C_{L\alpha}^o(\alpha_w-\frac{\Gamma_0}{2U_\infty b})S$ da cui $C_L=C_{L\alpha}^o(\alpha_w-\frac{\Gamma_0}{2U_\infty b})$. Per definizione $C_{L\alpha}^w=\frac{\de C_L}{\de\alpha_w}=C_{L\alpha}^o\big(1-\frac 1{2U_\infty b}\frac{\de\Gamma_0}{\de\alpha_w}\big)$ in cui posso sostituire $\frac{\de\Gamma_0}{\de\alpha_w}=\frac{2U_\infty S}{\pi b}C_{L\alpha}^w$ dalla precedente, ottenendo $C_{L_\alpha}^w=C_{L\alpha}^o\big(1-\frac{C_{L\alpha}^w}{\pi\text{A\!R}}\big)$ da cui $C_{L\alpha}^w=C_{L\alpha}^o\frac{\text{A\!R}}{2+\text{A\!R}}$\\
	\textit{Oss:} la pendenza della curva $C_L^w$-$\alpha_w$ è minore di quella della curva $C_L^o$-$\alpha$, ma la differenza diventa meno evidente al crescere di A\!R.
	
	\section{Curva polare dell'ala}
	\underline{Def:} $C_D^w=C_D^o+\frac{C_L^2}{e\pi A\!R}$\qquad ala non svergolata con profili uguali $C_D^w=C_{D0}+k(C_L+C_{LM})^2+\frac{C_L^2}{\pi A\!R}\simeq C_{D0}+kC_L^2$\\
	\textit{Dim:} per la teoria della linea portante posso scrivere per un'ala generica $C_D=C_D^o+C_{Di}=C_D^o+\frac{C_L^2}{\mathrm e\pi A\!R}$, con $\mathrm e=\frac{1}{1+\delta}$. Per un'ala non svergolata con profili uguali $C_D=C_D^o+k(C_L-C_{LM})+\frac{C_L^2}{\pi A\!R}$, cioè la polare dipende, a parità di altri parametri, dall'allungamento alare. Per incidenze non troppo elevate può essere approssimata da una relazione quadratica $C_D=C_D^o+kC_L^2$ in cui $C_D^o$ è il coeff. di resistenza nella condizione di portanza (globale) nulla (dovuto principalmente alla resistenza d'attrito).
	
	\section{Distribuzione della portanza e stallo}
	La distribuzione della portanza 
	\begin{enumerate}\item{$C_L(y)$ costante}
	\item{$C_L(y)$ ellittico}
	\item{$C_L(y)$ concentrato alle estremità}\end{enumerate}
	
	\section{Realizzare una distribuzione ellittica}
	\underline{Def:} \\
	\textit{Dim:} da (\ref{distr ell}) ottengo $\delta L(y)=\frac 12\rho U_\infty^2c(y)C_{L,\alpha}(y)\big(\alpha(y)-\alpha_\mathrm{ind}\big)\simeq\frac 12\rho U_\infty^2c(y)C_{L,\alpha}\big(\alpha(y)-\alpha_\mathrm{ind}\big)$. Per avere una distribuzione ellittica di questa posso fissare $\alpha(y)=\bar\alpha$ (cioè un'ala a svergolamento nullo) e considerare una distribuzione ellittica di $c(y)$, oppure posso fissare $c(y)=\bar c$ e distribuire opportunamente lo svergolamento. Tuttavia nella pratica, a causa di difficoltà costruttive, si impiegano ali rettangolari rastremate opportunamente. [DISTRIBUZIONE DI CL E CL C].
	
	\section{Winglets}
		Sono dispositivi posti alle estremità delle ali per diminuire il flusso trasversale e quindi la resistenza indotta. Per costruzione, sono soggette alla combinazione del flusso indisturbato e del flusso trasversale. Per questo la forza aerodinamica risultante ha una componente di spinta, con un effetto benefico, che dipende fortemente dal design della winglets, in particolare dall'efficienza della stessa e dall'intensità del flusso trasversale (quindi dall'incidenza a cui lavora.
	
	\section{Modalità di stallo}
	Anche se per un ala svergolata a profili uguali $L\propto C_L(y)c(y)$, la distribuzione di $C_L(y)$ influenza il comportamento allo stallo. Uno stallo di radice è favorito da una distribuzione con bassi $C_L$ in prossimità della fusoliera. Uno stallo di estremità è favorito da una distribuzione di $C_L$ decrescente verso l'estremità alare. Quest ultimo è più pericoloso perché di solito coinvolge la zona dell'ala in cui si trovano le superfici di controllo, e l'azione del flusso trasversale tende a propagare lo stallo su tutta l'ala. Bisogna tenere presente che la distribuzione di $C_L$ non deve essere costante in apertura per evitare lo stallo contemporaneo di tutti i profili dell'ala.

%%%%%%%%%%%%%%%%%%%%%%%%%%%%%%%%%%%%%%%%%%%%%%%%%%%%%%%%%%%%%%%%%%%%%%%%%%%%%%%%%%%%%%%%%%%%%%%%%%%%%%%%%%%%%%%%%%%%%%%%%%%%%%%%%%%%%%%%%%%%%%%%%%%%%%%%%%%%%%%%%%%%%%%%%%%%%%%%%%%%%%%%%%%%%%%%%%%%%%%%%%%%%%%%%%%%%%%%%%%%%%%%%%%%%%%%%%%%%%%%%%%%%%%%%%%%%%%%%%%%%%%%%%%%%%%%%%%%%%%%%%%%%%%%%%%%%%%%%%%%%%%%%%%%%%%%%%%%%%%%%%%%%%%%%%%%%%%%%%%%%%%%%%%%%%%%%%%%%%%%%%%%%%%%%%%%%%%%%%%%%%%%%%%%%%%%%%%%%%%%%%%%%%%%%%%%%%%%%%%%%%%%%%%%%%%%%%%%%%%%%%%%%%%%%%%%%%%%%%%%%%	

\chapter{Turbolenza}
	\section{Regime turbolento}
	\underline{Def:} $\vec V(P,t)=\langle\vec V(P)\rangle+\vec V^\prime(P,t)\quad p(P,t)=\langle p(P)\rangle+p^\prime(P,t)$\\
	\textit{Oss:} poiché è presente un termine aleatorio nelle grandezza coinvonte, i flussi turbolenti sono sempre non stazionari e tridimensionali. Inoltre piccolissime perturbazioni delle condizioni iniziali possono portare a dinamiche del flusso completamente diverse.
		\subsection{Cascata di energia}
		Sono presenti scambi di q.d.m. ed energia anche a livello macroscopico, i quali avvengono a scale anche più piccole delle dimensioni geometriche caratteristiche del problema considerato.  Questo è causato dal fatto che le strutture vorticose più grandi sono soggette ad instabilità che porta allaloro rottura in strutture vorticose più piccole. Il meccanismo si ripete finché le dimensioni sono tali da consentire agli effetti viscosi (\ref{vortex}) di dissipare tali strutture.

	\section{Sviluppo e criterio di transizione}
	Per flussi ad alto Re il regime laminare nello SL visto sin ora diventa instabile alle perturbazioni. Quello che succede è che dopo una regione di transizione si giunge in un regime di flusso turbolento. Per la lastra piana l'andamento tipico del regime nello SL consiste in un tratto laminare con inizio al BA e fine nel punto di transizione, in cui ha inizio una piccola regione di transizione in cui si verificano fenomeni complessi che conducono alla turbolenza. Poco più a valle del p.to di transizione inizia lo SL turbolento in cui le fluttuazioni aleatorie delle grandezze causano una tridimensionalizzazione del flusso. Un criterio usato per determinare la transizione è basato sul n. di Reylonds locale riferito alla distanza $x$ dal BA, cioè $\mathrm{Re}_x=\frac{Ux}{\nu}$. Tale criterio stabilisce un Re critico, riferito alla distanza $x_\mathrm{trans}$ del p.to di transizione, oltre cui si ha turbolenza $\mathrm{Re}_{cr}\equiv\mathrm{Re}_{x_\mathrm{trans}}=\frac{Ux_\mathrm{trans}}\nu$. Per valori $\mathrm{Re}_x<\mathrm{Re}_{cr}$ il regime è laminare, altrimenti diventa turbolento. Dalla relazione precedente si può ricavare $x_\mathrm{trans}=\frac{\nu}{U}\mathrm{Re}_{cr}$. Per la lastra piana $\mathrm{Re}_{cr}=4.5\cdot 10^5$ supponendo che le pareti siano perfettamente liscie e il flusso perfettamente laminare.
		\subsection{Parametri che influenzano la transizione}
		La rugosità influenza pesantemente la transizione alla turbolenza (vd. diagrammi di Moody), in particolare pareti ruvide hanno l'effetto di anticipare la transizione verso Re minori.\\
		Se il flusso ha già una aliquota di turbolenza, questa ha l'effetto di anticipare la transizione verso Re minori.\\
		Un gradiente favorevole di pressione l'effetto di ritardare la transizione verso Re maggiori mentre un gradiente avverso ha l'effetto di anticiparla.
		\subsection{Transizione nei profili alari}
		Per i profili la transizione avviene in un punto la cui coordinata può essere trovata tramite esperimenti ad hoc oppure analisi molto precise. In alternativa si può assumere lo stesso valore di $\mathrm{Re}_{cr}$ della lastra piana, sia sul dorso che sul ventre, con il quale determinare $x_\mathrm{trans}$. Se il profile è inclinato la transizione sul dorso e sul ventre non avviene alla stessa coordinata.

	\section{Strato limite turbolento}
	La differenza principale dallo SL laminare è che le particelle fluide, a causa delle fluttuazione di velocità, possono passare da uno strato fluido all'altro (strati caratterizzati da diversa velocità media). Le particelle che si spostano verso strati più vicini alla parete tendono a far aumentare la velocità  media di questi ultimi, viceversa le particelle che si muovono verso strati più lontani hanno l'effetto opposto. Risulta quindi uno scambio di q.d.m. fra le particelle che causa un rimescolamento delle stesse all'interno dello SL. Tuttavia in media si verifica in misura maggiore l'aumento di velocità degli strati più vicini alla parete. Questo causa un profilo di velocità con tangente $\frac{\partial u}{\partial y}$ più inclinata e quindi delle azioni viscose alla parete più elevate. Questo fatto implica che la resistenza d'attrito in presenza di turbolenza è maggiore. $\tau(x)=\mu\frac{\partial u}{\partial y}\vert_{y=0}$
%\begin{center}\begin{tabular}{cc}	\includegraphics[scale=.5]{profili-tbl.png}&\includegraphics[scale=.7]{tang.png}\\Profili di velocità&Tangente alla parete\end{tabular}\end{center}

%%%%%%%%%%%%%%%%%%%%%%%%%%%%%%%%%%%%%%%%%%%%%%%%%%%%%%%%%%%%%%%%%%%%%%%%%%%%%%%%%%%%%%%%%%%%%%%%%%%%%%%%%%%%%%%%%%%%%%%%%%%%%%%%%%%%%%%%%%%%%%%%%%%%%%%%%%%%%%%%%%%%%%%%%%%%%%%%%%%%%%%%%%%%%%%%%%%%%%%%%%%%%%%%%%%%%%%%%%%%%%%%%%%%%%%%%%%%%%%%%%%%%%%%%%%%%%%%%%%%%%%%%%%%%%%%%%%%%%%%%%%%%%%%%%%%%%%%%%%%%%%%%%%%%%%%%%%%%%%%%%%%%%%%%%%%%%%%%%%%%%%%%%%%%%%%%%%%%%%%%%%%%%%%%%%%%%%%%%%%%%%%%%%%%%%%%%%%%%%%%%%%%%%%%%%%%%%%%%%%%%%%%%%%%%%%%%%%%%%%%%%%%%%%%%%%%%%%%%%%%%	

\chapter{Flussi comprimibili}
	\section{Regimi di flusso}
	\underline{Def:} $0.3<\ma_\infty<\ma_\mathrm{cri}$\quad alto subsonico\\
$\ma_\mathrm{cri}<\ma_\infty<\ma_\mathrm{crs}$\quad transonico\\
$\ma_\mathrm{crs}<\ma_\infty<5$\quad supersonico\\
$\ma_\infty>5$\quad ipersonico\\
Per $\ma_\infty=\ma_\mathrm{cri}$ nel campo subsonico si raggiunge la condizione sonica in almeno un punto; per i profili tipicamente $\ma_\mathrm{cri}\approx 0.7\div 0.8$. Per $\ma_\infty=\ma_\mathrm{crs}$ nel campo supersonico si raggiunge la condizione sonica in almeno un punto; per i profili tipicamente $\ma_\mathrm{crs}\approx 1.2\div 1.3$.\\
	\textit{Oss:} in alto subsonico non si possono trascurare gli effetti della comprimibilità ma in nessun punto del campo di moto di raggiungono le condizioni soniche. Se in qualche punto del campo vengono raggiunte tali condizioni allora si è in transonico. 
	
	\section{Propogazione di un disturbo infinitesimo}
	Considero una sorgente puntiforme che emette una piccolissima perturbazione di pressione nel fluido. Essa si propaga isotropicamente con la velocità del suono nel fluido considerato.  Inizialmente la sorgente è ferma e il disturbo si propaga secondo sfere cencentriche centrate nella sorgente.\\Ora la sorgente è in moto rettilineo uniforme rispetto al fluido con velocità $V<a$. Il disturbo si propaga secondo sfere centrate nella sorgente in moto e che quindi non sono più concentriche ma addensate nella direzione del moto. La sorgente rimane sempre all'interno di regioni del campo in cui il disturbo si è già propagato.\\Ora la velocità della sorgente è $V=a$. Il disturbo si propaga secondo sfere \sz{(centrate nella sorgente in moto)} che sono tangenti fra di loro e tangenti ad una piano normale alla direzione della sorgente e passante per la sorgente stessa. Il disturbo riesce a propagarsi solo a valle della sorgente.\\Ora la velocità della sorgente è $V>a$. Il disturbo si propaga con le stesse modalità ma la sorgente è sempre in regioni del campo in cui il disturbo (emesso negli istanti precedenti) non si è ancora propagato.  Le sfere sono tutte tangenti ad un cono (c. di Mach) con vertice nella sorgente, che individua un diedro (d. di Mach) entro cui sono confinati i disturbi mentre all'esterno il flusso è indisturbato. L'angolo $\beta$ di semiapertura del diedro (a. di Mach) è dato da $\sin\beta=\frac{a\delta t}{V\delta t}=\frac 1\ma$


\end{document}
