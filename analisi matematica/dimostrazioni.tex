\documentclass[10pt,a4paper]{article}

%% PACCHETTI AGGIUNTIVI
\usepackage{amssymb,amsmath,amsthm,amsfonts}
%\usepackage{bm}
\usepackage{calc}
%\usepackage[inline]{enumitem}
%\usepackage{ifthen}
%\usepackage[utf8]{inputenc}
\usepackage[portrait]{geometry}
%\usepackage{graphicx}
%\usepackage[colorlinks=true,citecolor=blue,linkcolor=blue]{hyperref}
\usepackage{mathrsfs}
%\usepackage{multicol,multirow}
%\usepackage{subcaption}
%\usepackage{tabularx}
%\usepackage[absolute]{textpos}
%\usepackage{titlesec}
%\usepackage{wrapfig}
%\usepackage{xfrac}


%% GEOMETRIA
\geometry{top=1cm,bottom=1cm,left=.7cm,right=.7cm}
	
%%	STILE
\pagestyle{empty}
%\raggedright

%%	HEADINGS
%%		Numerazione
\setcounter{secnumdepth}{0}
\setlength{\parindent}{3pt}
\setlength{\parskip}{0pt plus 0.5ex}

%%		Formattazione capitoli
%\titleformat{\chapter}[hang]{\normalfont\huge\bfseries}{\thechapter\quad}{0cm}{}  	% hang, block, display, runin

%%		Formattazione headings \titlesec {spazio-sx}{spazio-prima}{spazio-dopo}
%\titlespacing*{\section}{-1.5ex}{1ex}{0ex}
%\titlespacing*{\subsection}{0ex}{.3ex}{0ex}
%\titlespacing*{\subsubsection}{0ex}{0ex}{0ex}

%%		Altra formattazione
%\makeatletter
%\renewcommand{\section}{\@startsection{section}{1}{-3mm}{2ex}{.1ex}{\normalfont\large\bfseries}}
%\renewcommand{\subsection}{\@startsection{subsection}{2}{0mm}{.5ex}{.1ex}{\normalfont\normalsize\bfseries}}
%\renewcommand{\subsubsection}{\@startsection{subsubsection}{3}{1mm}{.1ex}{.1ex}{\normalfont\small\bfseries}}
%\makeatother

%%	DEFINIZIONE COMANDI
\newcommand{\de}{\mathrm d}
\newcommand{\fracd}[2]{\frac{\de #1}{\de #2}}
\newcommand{\fracp}[2]{\frac{\partial #1}{\partial #2}}
\newcommand{\fracpq}[2]{\frac{\partial^2 #1}{{\partial #2}^2}}
\newcommand{\fracpp}[3]{\frac{\partial^2 #1}{\partial #2 \partial #3}}
\newcommand{\grad}[1]{\text{grad}\,#1}
\newcommand{\dive}[1]{\text{div}\,#1}
\newcommand{\rot}[1]{\text{rot}\,#1}
\newcommand{\vers}{\mathop{\text{vers}}}
\newcommand{\itemm}[1]{\indent - #1\\}
\newcommand{\tr}[1]{\text{tr}\,#1}
\newcommand{\sym}[1]{\text{sym}\,#1}
\newcommand{\skw}[1]{\text{skw}\,#1}
\newcommand{\sz}[1]{\scriptsize #1\normalsize}
\newcommand{\mach}{\text{Ma}}


\begin{document}
%\tableofcontents\newpage
\subsection{Il limite preserva le operazioni algebriche}
$f_1 \longrightarrow L_1$ e $f_2 \longrightarrow L_2$ per $x \longrightarrow x_0 \Longrightarrow$ $f_1+f_2 \longrightarrow L_1+L_2$\\
Dimostrazione: $\forall \epsilon\;\exists \delta_1,\delta_2$ tale che $|x-x_0|<\delta_1 \Rightarrow |f_1(x)-L_1|<\epsilon$ e $|x-x_0|>\delta_2 \Rightarrow |f_2(x)-L_2|<\epsilon$. Se prendo $\delta=\min\{\delta_1,\delta_2\}$ allora $|x-x_0|<\delta$ implica sia $|f_1(x)-L_1|<\epsilon$ che $ |f_2(x)-L_2|<\epsilon$. Allora sommando membro a membro $|f_1-L_1| + |f_2-L_2|<2\epsilon$ e per la disuguaglianza triangolare $|f_1+f_2-L_1-L_2|<2\epsilon$ da cui la tesi.\\

$f_1 \longrightarrow L_1$ e $f_2 \longrightarrow L_2$ per $x \longrightarrow x_0 \Longrightarrow f_1 f_2 \longrightarrow L_1 L_2$\\

$f_1 \longrightarrow L_1$ e $f_2 \longrightarrow L_2$ per $x \longrightarrow x_0 \Longrightarrow \frac{f_1}{f_2} \longrightarrow \frac{L_1} {L_2}$

\subsection{Teorema della permanenza del segno}
 $\lim_{x\to x_0} f(x)>0 \Rightarrow \exists B(x_0)$ in cui $\forall x\in B(x_0)$ si ha $f(x)>0$\\
 Dimostrazione: sia $L=\lim_{x\to x_0} f(x)$ e scelgo $\epsilon<L/2$. Per ipotesi $\exists \delta>0$ tale che $|x-x_0|<\delta \Rightarrow |f(x)-L|<L/2$. Ma $|f(x)-L|<L/2$ vuol dire $L-L/2 < f(x) < L+L/2$. Dato che $L>0$, dalla prima disuguaglianza segue $f(x)>L/2>0$
 
 
\end{document}